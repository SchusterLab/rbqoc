\documentclass[letterpaper, 12pt]{article}

% imports
\usepackage{appendix}[titletoc]
\usepackage[english]{babel}
\usepackage[justification=centering]{caption}
\usepackage{float}
\usepackage{graphicx}
\usepackage{hyperref}
\usepackage[utf8]{inputenc}
\usepackage{layouts}
\usepackage{optidef}
\usepackage{physics}
\usepackage{setspace}
\usepackage[labelformat=simple]{subcaption}
\usepackage{xcolor}

% configure imports
\DeclareMathOperator*{\argmin}{arg\,min}
\newcommand{\todo}[1]{\textcolor{red}{TODO: #1}}
\definecolor{darkgreen}{RGB}{46, 184, 46}
\newcommand{\half}{\frac{1}{2}}
\newcommand{\R}{\mathbb{R}}
\captionsetup[subfigure]{labelformat=empty, skip=0pt}
\restylefloat{table}
\renewcommand\thesubfigure{(\alph{subfigure})}



\begin{document}

\begin{center}
  {\huge Schuster Lab - Autumn 2019 Report} \\[0.5em]
  {\large Thomas Propson $\vert$ \href{mailto:tcpropson@uchicago.edu}
    {tcpropson@uchicago.edu} \\[0.5em] \today}
\end{center}

\section{Simulation Parameters}
\subsection{Spin System}
Dave told me these system parameters on 08/25/2019. The system hamiltonian is
\begin{align*}
  H &= \omega_{q}\frac{\sigma_{z}}{2}\\
    &+ \epsilon(t) \frac{\sigma_{x}}{2}\\
\end{align*}
$\omega_{q} = 1 * 10^{-2}$GHz, $\epsilon(t) \in \mathbb{R}, |\epsilon(t)| \le 1 * 10^{-1}$GHz.
The sampling rate of the AWG is 1.2 Gs/s with 14 bits. The bandwidth of the pulses should be
less than $5 * 10^{-1}$GHz. We choose the duration of the pulse to be about $150$ns. 

\section{Hamiltonian Parameter Robustness}
We want to design pulses that are robust to variaitions
in the hamiltonian parameter $\omega_{q}$. We seek to find pulses that minimize the
norm of the derivative of $\omega_{q}$ w.r.t the derivative of $\epsilon(t)$ w.r.t to the gate fidelity.
We implement the universal gate set $\{R_{x}(\pi), R_{x}(\frac{\pi}{2}), R_{y}(\frac{\pi}{2}), T\}$
\cite{heeres2017implementing}.

\subsection{Spin Experiment 0}
This experiment achieves the $R_{x}(\pi)$ gate
  \[
    \begin{pmatrix}
      1\\
      0\\
    \end{pmatrix}
    \rightarrow
    \begin{pmatrix}
      0\\
      -i\\
    \end{pmatrix}
   \]
   \[
    \begin{pmatrix}
      0\\
      1\\
    \end{pmatrix}
    \rightarrow
    \begin{pmatrix}
      -i\\
      0\\
    \end{pmatrix}
  \]

  \subsection{Spin Experiment 1}
  This experiment achieves the $R_{x}(\frac{\pi}{2})$ gate
  \[
    \begin{pmatrix}
      1\\
      0\\
    \end{pmatrix}
    \rightarrow
    \begin{pmatrix}
      \frac{\sqrt{2}}{2}\\
      -i\frac{\sqrt{2}}{2}\\
    \end{pmatrix}
   \]
   \[
    \begin{pmatrix}
      0\\
      1\\
    \end{pmatrix}
    \rightarrow
    \begin{pmatrix}
      -i\frac{\sqrt{2}}{2}\\
      \frac{\sqrt{2}}{2}\\
    \end{pmatrix}
  \]

  \subsection{Spin Experiment 2}
  This experiment achieves the $R_{y}(\frac{\pi}{2})$ gate
  \[
    \begin{pmatrix}
      1\\
      0\\
    \end{pmatrix}
    \rightarrow
    \begin{pmatrix}
      \frac{\sqrt{2}}{2}\\
      \frac{\sqrt{2}}{2}\\
    \end{pmatrix}
   \]
   \[
    \begin{pmatrix}
      0\\
      1\\
    \end{pmatrix}
    \rightarrow
    \begin{pmatrix}
      -\frac{\sqrt{2}}{2}\\
      \frac{\sqrt{2}}{2}\\
    \end{pmatrix}
  \]

  \subsection{Spin Experiment 3}
  This experiment achieves the $T$ gate
  \[
    \begin{pmatrix}
      1\\
      0\\
    \end{pmatrix}
    \rightarrow
    \begin{pmatrix}
      1\\
      0\\
    \end{pmatrix}
\]
\[
    \begin{pmatrix}
      0\\
      1\\
    \end{pmatrix}
    \rightarrow
    \begin{pmatrix}
      0\\
      \frac{\sqrt{2}}{2} + i\frac{\sqrt{2}}{2}\\
    \end{pmatrix}
 \]

  \subsection{Spin Experiment 4,5,6, pi\_corpse}
  These experiments were intended to build a CORPSE $R_{x}(\pi)$ pulse
  following this Ken Brown paper \cite{merrill2014progress}.


  \subsection{Spin Experiment 7}
  This experiment is the same as experiment 0 with the added constraint of
  zero integration.


  \section{Amplitude Robustness}

  \subsection{Cavity Experiment 0}
  0 to 2 in the cavity following \cite{heeres2017implementing}.
  
  

\bibliography{report/refs}
\bibliographystyle{plainnat}

\end{document}
