\section{Conclusion}
In conclusion, we have demonstrated techniques for achieving robustness to systematic
errors and mitigating decoherence on a quantum system using state-of-the-art trajectory
optimization. We have proposed a scheme for mitigating longitudinal relaxation with time-optimal
control and an efficient optimization metric that comes at a constant computational cost as
opposed to integrating a master equation which scales quadratically with
the dimension of the Hilbert space.
We have proposed the derivative method for robust control which achieves
super-linear gate error reductions in the gate duration for the problem we studied here.
We have shown that robust control techniques can be used to mitigate decoherence due
to 1/$f$ flux noise, a dominant source of coherent errors for flux controlled qubits.
The numerical techniques we have studied here can be used to perform phase gates in arbitrary times,
which will be critical for synchronizing multi-qubit systems. Additionally,
interleaving the error models with existing
characterization methods will improve their effectiveness in experiments.
These techniques will be employed to achieve the low gate errors
required for fault-tolerant quantum computing applications. Our
implementation of the techniques described in this work is available
at \url{https://github.com/SchusterLab/rbqoc}.
