\section{Conclusion}
We have applied state-of-the-art trajectory
optimization techniques to mitigate decoherence and
achieve robustness to parameter uncertainty
errors on a quantum system.
We have proposed a scheme for suppressing
depolarization with time-optimal
control and the integrated depolarization rate model.
The computational cost of this model is
independent of the dimension of the Hilbert space, enabling
inexpensive optimization on high-dimensional quantum systems (hmm, this seems like something that definitely wasn't emphasized in the text - could be worth talking about this a little earlier. Also doesn't seem totally true right? Re Eq.(5)).
We have also proposed the derivative method for robust control which achieves
super-linear gate error reductions in the gate time for the static parameter
uncertainty problem we studied.
We have shown that the derivative, sampling, and unscented sampling methods
can mitigate 1/$f$ flux noise errors--which
dominate coherent errors for flux controlled qubits.
These robust control techniques can be applied
to any Hamiltonian,
allowing experimentalists in all domains to engineer robust
operations on their quantum systems.
These methods can be used to achieve the low gate errors
required for fault-tolerant quantum computing applications. Our
implementations of the techniques described in this work are available
at \url{https://github.com/SchusterLab/rbqoc}. \todo{static code?}
