\section{Conclusion}
In conclusion, we have applied state-of-the-art trajectory
optimization techniques to mitigate decoherence and
achieve robustness to parameter uncertainty
errors on a quantum system.
We have proposed a scheme for supressing
depolarization with time-optimal
control and the depolarization probability model.
The computational cost of this depolarization model is
independent of the dimension of the Hilbert space, enabling
inexpensive optimization on high-dimensional quantum systems.
We have also proposed the derivative method for robust control which achieves
super-linear gate error reductions in the gate time for the static parameter
uncertainty problem we studied.
We have shown that the derivative, sampling, and unscented sampling methods
can mitigate 1/$f$ flux noise errors
which dominate coherent errors for flux controlled qubits.
\myrem{The derivative and sampling methods scale cubically with
the dimension of the Hilbert space, allowing for applications
to larger quantum systems.
The derivative, sampling, and unscented sampling
methods will benefit from interleaving optimization with
experimental characterization, while the sampling and unscented
sampling methods can also be applied to error channels without models.}
\myadd{The robust control techniques can be applied
to any Hamiltonian,
allowing experimentalists in all domains to engineer robust
operations on their quantum systems.}
These methods will be used to achieve the low gate errors
required for fault-tolerant quantum computing applications. Our
implementation of the techniques described in this work is available
at \url{https://github.com/SchusterLab/rbqoc}.
