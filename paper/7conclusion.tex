\section{Conclusion}
We have introduced state-of-the-art trajectory optimization
techniques in the context of quantum optimal control, enabling
us to achieve tight tolerances for multiple constraints on the
control fields and quantum states. Using these capabilities,
we have mitigated decoherence and
achieved robustness to parameter uncertainty
errors on a superconducting fluxonium qubit.
We have proposed a scheme for suppressing
depolarization with time-optimal
control and the integrated depolarization rate model.
The computational complexity of evaluating this model is
independent of the dimension of the Hilbert space, enabling
inexpensive optimization on high-dimensional quantum systems.
We have also proposed the derivative method for robust control which achieves
superlinear gate error reductions in the gate time for the static parameter
uncertainty problem we studied.
We have shown that the derivative, sampling, and unscented sampling methods
can mitigate 1/$f$ flux noise errors -- which
dominate coherent errors for flux controlled qubits.
These robust control techniques can be applied
to any Hamiltonian,
allowing experimentalists in all domains to engineer robust
operations on their quantum systems.
Furthermore, they can be used to achieve the low gate errors
required for fault-tolerant quantum computing applications. Our
implementations of the techniques described in this work are available
at \url{https://github.com/SchusterLab/rbqoc}.

