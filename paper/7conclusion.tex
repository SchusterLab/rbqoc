\section{Conclusion}
In conclusion, we have applied state-of-the-art trajectory
optimization techniques to achieve robustness to parameter deviations
and mitigate decoherence on a quantum system.
We have proposed a scheme for mitigating
longitudinal relaxation with time-optimal
control and the probability of longitudinal relaxation metric,
avoiding the optimal control problem for density matrices which scales quartically with
the dimension of the Hilbert space.
We have proposed the derivative method for robust control which achieves
super-linear gate error reductions in its gate time for the static parameter
deviation problem we studied.
We have shown that the derivative, sampling, and unscented sampling methods
can mitigate 1/$f$ flux noise errors
which dominate coherent errors for flux controlled qubits.
The derivative and sampling methods scale cubically with
the dimension of the Hilbert space, allowing for applications
to larger quantum systems.
The derivative, sampling, and unscented sampling
methods will benefit from interleaving optimization with
experimental characterization, while the sampling and unscented
sampling methods can also be applied to error channels without models.
These techniques will be used to achieve the low gate errors
required for fault-tolerant quantum computing applications. Our
implementation of the techniques described in this work is available
at \url{https://github.com/SchusterLab/rbqoc}.
