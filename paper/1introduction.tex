\section{Introduction}
Quantum optimal control (QOC) is a class of optimization
algorithms for accurately and efficiently manipulating quantum systems.
Early techniques were proposed for nuclear magnetic resonance experiments
\cite{vandersypen2005nmr, kehlet2004improving, khaneja2005optimal,
  maximov2008optimal, nielsen2010optimal, skinner2003application, tosner2009optimal},
and applications now include superconducting circuits \cite{abdelhafez2020universal,
  chakram2020multimode, fisher2010optimal, gokhale2019partial,
  huang2020engineering, leng2019robust, leung2017speedup, li2020fast,
  xu2020nonadiabatic},
neutral atoms and ions \cite{brouzos2015quantum,
  de2008optimal, goerz2011quantum, guo2019high, jensen2019time,
  larrouy2020fast, omran2019generation, rosi2013fast, sorensen2019qengine,
  treutlein2006microwave, van2016optimal},
nitrogen-vacancy centers in diamond \cite{chou2015optimal,
  dolde2014high, geng2016experimental,
  nobauer2015smooth, poggiali2018optimal, rembold2020introduction, tian2019optimal},
and Bose-Einstein condensates \cite{amri2019optimal, sorensen2018quantum}.
For quantum computation,
optimal control is employed to achieve high-fidelity gates
while adhering to experimental constraints.
Experimental errors may cause the system to deviate
from the model used in optimization, leading
to poor experimental performance.
Robust control improves upon
standard optimal control by encoding
model parameter uncertainties
in optimization objectives, yielding performance
guarantees over a range of parameter values \cite{Zhou97,Morimoto00,Manchester18}.
We adapt robust control techniques from the robotics community to mitigate
parameter uncertainty errors for
a superconducting fluxonium qubit.

QOC has had tremendous success in
engineering high-fidelity quantum gates \cite{
  chou2015optimal, dolde2014high, egger2013optimized, egger2014adaptive,
  grace2007optimal, heeres2017implementing, huang2014optimal,
  kelly2014optimal, leng2019robust, liebermann2016optimal,
  nebendahl2009optimal, rebentrost2009optimal, rebentrost2009optimal2,
  spiteri2018quantum, sporl2007optimal}. 
However, the system often
deviates from the model used in optimization
due to experimental errors such as parameter drift, noise, finite control 
resolution, and decoherence. Multiple techniques have been
developed to address this shortcoming.
Analytically-derived control pulses that
mitigate parameter uncertainty errors include
composite pulses \cite{cummins2000use, cummins2003tackling,
  kupce1995stretched, merrill2014progress},
pulses designed by considering dynamic and geometric phases
\cite{han2020experimental, xu2020nonadiabatic}, and
pulses obtained with the DRAG scheme \cite{motzoi2009simple}.
To suppress decoherence, Floquet techniques have been employed
\cite{huang2020engineering, mundada2020floquet}.
Numerical schemes to mitigate parameter uncertainty errors
include closed-loop methods \cite{egger2014adaptive, feng2018gradient,
  wittler2020integrated} and open-loop methods \cite{
  allen2019robust, carvalho2020error, reinhold2019controlling,
  rembold2020introduction, kosut2013robust}.
To model decoherence, numerical techniques typically employ
master equations \cite{rembold2020introduction, schulteherbruggen2011optimal} or
Monte Carlo-style quantum trajectories \cite{abdelhafez2019gradient}.

In this work, we study three robust control techniques that
make the system's quantum state trajectory less sensitive
to static and time-dependent parameter uncertainty:
\begin{enumerate}
  \item A sampling method, similar to the work of \cite{allen2019robust,
    carvalho2020error, khaneja2005optimal, reinhold2019controlling, rembold2020introduction}.
  \item An unscented sampling method adapted from the unscented transform used in
    state estimation \cite{howell2020direct, julier2004unscented,
      lee2013sigma, manchester2016derivative}.
  \item A derivative method, which penalizes the sensitivity of the quantum state trajectory
    to uncertain parameters.
\end{enumerate}
We apply these techniques to the fluxonium qubit presented in \cite{zhang2020universal}.
We also show that QOC can solve important problems associated with
fluxonium-based qubits: mitigating
depolarization by taking advantage of the $T_{1}$-dependence of the controls
and synchronizing qubits with distinct frequencies
by performing phase gates in arbitrary times.
To mitigate depolarization,
we perform time-optimal control and
employ an efficient depolarization model that 
is significantly less expensive to compute than integrating a master equation.
Leveraging recent advances in trajectory optimization within the field of robotics, we
solve these optimization problems using ALTRO (Augmented Lagrangian TRajectory Optimizer)
\cite{howell2019altro}, which uses the iterative linear quadratic regulator (iLQR) method
\cite{Li2004a} within an augmented Lagrangian framework \cite{lantoine2012hybrid,
  plancher2017constrained}
to handle nonlinear equality and inequality constraints at 
each time step.

This paper is organized as follows.
First, we reintroduce ALTRO in the context of QOC
in Section \ref{sec:background}.
We describe realistic constraints for operating the fluxonium and
define the associated QOC problem in Section \ref{sec:fluxonium}.
Then, we outline a method for making the optimization aware of depolarization
in Section \ref{sec:longitude}. Next, we outline three techniques for achieving
robustness to static parameter uncertainties in Section \ref{sec:static}. We
adapt the same techniques to mitigate 1/$f$ flux noise
in Section \ref{sec:stochastic}.
