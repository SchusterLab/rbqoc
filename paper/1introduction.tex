\section{Introduction}
Quantum optimal control (QOC) is a class of optimization
algorithms for accurately and efficiently manipulating quantum systems.
Early techniques were proposed for nuclear magnetic resonance experiments
\cite{vandersypen2005nmr, kehlet2004improving, khaneja2005optimal,
  maximov2008optimal, nielsen2010optimal, skinner2003application, tosner2009optimal},
and applications now include superconducting circuits \cite{abdelhafez2020universal,
  chakram2020multimode, fisher2010optimal, gokhale2019partial,
  leng2019robust, leung2017speedup, li2020fast},
neutral atoms and ions \cite{brouzos2015quantum,
  de2008optimal, goerz2011quantum, guo2019high, jensen2019time,
  larrouy2020fast, omran2019generation, rosi2013fast, sorensen2019qengine,
  treutlein2006microwave, van2016optimal},
nitrogen-vacancy centers in diamond \cite{chou2015optimal,
  dolde2014high, geng2016experimental,
  nobauer2015smooth, poggiali2018optimal, rembold2020introduction, tian2019optimal},
and Bose-Einstein condensates \cite{amri2019optimal, sorensen2018quantum}.
In the context of quantum computation,
optimal control is employed to achieve high-fidelity gates
while adhering to experimental constraints.
Experimental errors such as parameter drift, noise, and
finite control resolution cause the system to deviate
from the model used in optimization, leading
to poor experimental performance (could be a good place for a citation here? It obviosuly makes sense in principle, but I'm not aware of a paper that tried to implement optimal control and failed for these reasons. *on reading further* Maybe the Zhou book from below).
Robust control improves upon
standard optimal control by encoding
model parameter uncertainties
in optimization objectives, yielding performance
guarantees over a range of parameter values \cite{Zhou97,Morimoto00,Manchester18}.
We adapt robust control techniques from the robotics community to mitigate
parameter uncertainty errors for
a superconducting fluxonium qubit.

Analytically-dervied control pulses that mitigate parameter uncertainty
errors include composite pulses \cite{cummins2000use, cummins2003tackling,
  kupce1995stretched, merrill2014progress},
pulses designed by considering dynamic and geometric phases
\cite{han2020experimental, xu2020nonadiabatic}, and
pulses obtained with the DRAG scheme \cite{motzoi2009simple}.
These techniques are not amenable to arbitrary experimental constraints,
so recent work has sought to achieve robustness in quantum optimal
control frameworks using closed-loop methods \cite{egger2014adaptive, feng2018gradient,
  wittler2020integrated} and open-loop methods \cite{
  allen2019robust, carvalho2020error, reinhold2019controlling,
  rembold2020introduction, kosut2013robust}.

In this work, we study three robust control techniques that
make the system's quantum state trajectory less sensitive
to static and time-dependent parameter uncertainty:
\begin{enumerate}
\item A sampling method, similar to the work of \cite{allen2019robust,
  ball2020software, carvalho2020error, khaneja2005optimal,
  reinhold2019controlling, rembold2020introduction}.
  \item An unscented sampling method adapted from the unscented transform used in
    state estimation \cite{howell2020direct, julier2004unscented,
      lee2013sigma, thangavel2020robust}.
  \item A derivative method, which penalizes the sensitivity of the quantum state trajectory
    to uncertain parameters.
\end{enumerate}
We apply these techniques to the fluxonium qubit presented in \cite{zhang2020universal}.
We also show that QOC can solve important problems associated with
fluxonium-based qubits: mitigating
depolarization by taking advantage of the $T_{1}$-dependence of the controls
and synchronizing qubits with distinct frequencies
by performing phase gates in arbitrary times.
To mitigate depolarization,
we perform time-optimal control and
employ an efficient depolarization model
for which the computational cost is independent of the
Hilbert space dimension.
Leveraging recent advances in trajectory optimization within the field of robotics, we
solve these optimization problems using ALTRO (Augmented Lagrangian TRajectory Optimizer)
\cite{howell2019altro}, which can enforce constraints on the quantum state trajectory
and control parameters.

This paper is organized as follows.
First, we describe ALTRO in the context of QOC
in Section \ref{sec:background}.
We outline realistic constraints for operating the fluxonium and
define the associated QOC problem in Section \ref{sec:fluxonium}.
Then, we formulate a method for supressing depolarization
in Section \ref{sec:longitude}. Next, we describe three techniques for achieving
robustness to static parameter uncertainties in Section \ref{sec:static}. We
adapt the same techniques to mitigate 1/$f$ flux noise
in Section \ref{sec:stochastic}.
