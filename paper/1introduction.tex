%% S1
\section{Introduction}
%% textwidth: \printinunitsof{in}\prntlen{\textwidth}
%% linewidth: \printinunitsof{in}\prntlen{\linewidth}
%% P1 - general field, lots of citation. cite reviews, original qoc.
%% P2 - exhaustively find all papers, what people have done to address the issues
%% P3 - what we do is. maybe add a rant about how
%% QOC is classical control theory but we've ignored that field
%% P4 - outline
Quantum optimal control (QOC) techniques are a class of optimization
algorithms for accurately and efficiently manipulating quantum systems.
Early techniques were proposed for nuclear magnetic resonance experiments
\cite{khaneja2005optimal}, and applications now include superconducting
circuits \cite{heeres2017implementing,
  huang2020engineering, leng2019robust, leung2017speedup, xu2020nonadiabatic},
neutral and ionized atoms \cite{van2016optimal}, nitrogen-vacancy centers in
diamond \cite{rembold2020introduction}, and Bose-Einstien condensates
\cite{sorensen2018quantum}. QOC techniques aim to control the system
such that a set of objectives are optimally satisfied.
For quantum computation relevant objectives include achieving high fidelity
gates and adhering to experimental constraints.
The decision variables of the optimization problem are the time-dependent control
parameters, particular to the quantum system, that govern its evolution.
Experimental errors may cause the system evolution to deviate from that predicted in
optimization, hampering performance.
The field of classical control theory has developed robust control techniques
to encode experimental errors in optimization objectives, improving
experimental performance (citations appreciated).
In this work we employ robust control techniques to mitigate
releastic decoherence and systematic errors that arise when controlling
a superconducting fluxonium qubit.

Analytic and numerical techniques for QOC have been remarkably successful in
designing high fidelity gates (citations needed). In addition to the high
simulated fidelities which can be obtained with these techniques, it is desirable for the
resultant quantum state trajectory to be insensitive to experimental errors
such as parameter drift, noise, and finite control resolution. Controls that produce
quantum state trajectories that are insensitive to these effects are said to be robust.
In line with this objective, recent work has sought to mitigate systematic errors
and decoherence present in the experiment.
Considering the dynamical and geometric phases of quantum state
trajectories has led to analytic methods for achieving
robustness to systematic errors and mitigating pure dephasing
\cite{han2020experimental, merrill2014progress, xu2020nonadiabatic, zhang2020universal}.
Floquet techniques have been experimentally demonstrated to simultaneously mitigate
decoherence due to longitudinal relaxation and pure dephasing
\cite{huang2020engineering, mundada2020floquet}.
Numerical QOC techniques have been adapted to mitigate longitudinal relaxation
by modeling master equations \cite{rembold2020introduction} and employing
Monte Carlo style quantum trajectories \cite{abdelhafez2019gradient}.
Additionally, efforts have been made to incorporate experimental feedback
into optimization \cite{huang2020engineering}. Comprehensive tools that interleave characterization
and pulse design are under active development \cite{wittler2020integrated}.

In this work we employ state of the art trajectory optimization techniques
to mitigate systematic errors and decoherence. We frame the QOC problem as an
open loop optimization problem, optimizing the controls to make the
quantum state trajectory robust to systematic errors and decoherence.
We consider three robust trajectory optimization techniques.
The first is the sampling method, which has been applied
previously in the context of QOC
\cite{carvalho2020error, reinhold2019controlling, rembold2020introduction}.
The second is the unscented sampling
method, which derives from the unscented transformation used
for non-linear Kalman filtering
\cite{julier2004unscented, lee2013sigma, manchester2016derivative}.
We propose a third method, which uses derivative information
of parameter deviations with respect to the quantum state trajectory.
We make this method efficient by employing mixed-mode differentiation.
Additionally, we take advantage of the known dependence of the
system's controls on longitudinal relaxation. We encode
the decoherence in an efficient optimization objective that does
not pay the increased computational cost of integrating a master equation.
We perform these optimizations using the ALTRO method \cite{howell2019altro}.
Many popular numerical techniques
for QOC use gradient-based control update procedures, which are readily amenable to control filtering
or constraint manifold projection techniques to enforce constraints \cite{leung2017speedup,
  goer2019krotov, abdelhafez2019gradient, machnes2015gradient}.
The ALTRO method we employ combines
an iterative indirect shooting method with the augmented Lagrangian
method. The augmented Lagrangian method
allows us to enforce arbitrary, simultaneous constraints and achieve
fast convergence without being restricted to the constraint
manifold.

We compare the performance of these numerical techniques
on the heavy fluxonium qubit and against the assocaited analytic techniques
presented in \cite{zhang2020universal}. The fluxonium is an exciting
platform for quantum computation due to its long longitudinal relaxation
and pure dephasing times at the flux frustation point. This system is also
efficient to simulate due to its accurate two-level approximation.
We emphasize that the numerical techniques we employ extend to arbitrary quantum systems,
not only the one we study here. We describe experimentally realistic constraints for this system
and map them to the ALTRO framework. Next, we
outline a method for making the optimization aware of longitudinal
relaxation. We achieve a factor of 5 decrease in single gate errors due to
longitudinal relaxation over the analytic gate set. Next we outline three methods for achieving
robustness to static parameter deviations. We find that we are able to
decrease the gate error arising from static parameter deviations super-linearly
in the gate duration, achieving orders of magntidue reductions in single gate errors.
We also find that the numerical techniques we employ are able to produce
fixed phase gates in arbitrary times, a method that could be used
to control multi-qubit systems in the lab frame. Finally,
we employ the robust control techniques to mitigate time-varying
magnetic flux noise. We achieve a two order of magnitude improvement
in gate errors for extended computations, an improvement critical
for noisy, intermediate-scale quantum (NISQ) applications.
