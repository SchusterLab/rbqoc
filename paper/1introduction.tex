\section{Introduction}
Quantum optimal control (QOC) techniques are a class of optimization
algorithms for accurately and efficiently manipulating quantum systems.
Early techniques were proposed for nuclear magnetic resonance experiments
\cite{khaneja2005optimal}, and applications now include superconducting
circuits \cite{heeres2017implementing,
  huang2020engineering, leng2019robust, leung2017speedup, xu2020nonadiabatic},
neutral and ionized atoms \cite{van2016optimal}, nitrogen-vacancy centers in
diamond \cite{rembold2020introduction}, and Bose-Einstien condensates
\cite{sorensen2018quantum}.
For quantum computation
QOC techniques are employed to achieve high fidelity gates
while adhering to experimental constraints.
Experimental errors may cause the system's trajectory to deviate from that predicted in
optimization, hampering performance.
Robust control techniques 
address this issue by encoding
experimental errors in optimization objectives, yielding performance
garauntees in the presence of model uncertainty \cite{Zhou97,Morimoto00,Manchester18}.
In this work we employ robust control techniques to mitigate
realistic system parameter deviations and decoherence that arise when controlling
a superconducting fluxonium qubit.

Analytic and numerical techniques for QOC have been remarkably successful in
designing high fidelity gates \todo{citations needed}. 
While these methods are able to predict system behavior with high accuracy, they tend 
to be sensitive to experimental errors such as parameter drift, noise, and finite control 
resolution. 
In order to address these issues,
recent work has sought to mitigate
errors due to parameter deviations.
Considering the dynamical and geometric phases of quantum state
trajectories has led to analytic methods for mitigating
errors due to parameter deviations and pure dephasing
\cite{han2020experimental, merrill2014progress, xu2020nonadiabatic, zhang2020universal}.
Floquet techniques have been experimentally demonstrated to simultaneously mitigate
longitudinal relaxation and pure dephasing decoherence
\cite{huang2020engineering, mundada2020floquet}.
Numerical QOC techniques have been adapted to mitigate longitudinal relaxation
by modeling master equations \cite{rembold2020introduction} and employing
Monte Carlo style quantum trajectories \cite{abdelhafez2019gradient}.
Additionally, efforts have been made to incorporate experimental feedback
into optimization \cite{huang2020engineering}. Comprehensive tools that interleave characterization
and pulse design are under active development \cite{wittler2020integrated}.

In this work, we study three robust control techniques that
make the optimized trajectory less sensitive
to static and time-dependent parameter deviations:
\begin{enumerate}
\item A sampling method, similar to the work of \cite{carvalho2020error, reinhold2019controlling,
  rembold2020introduction}.
  \item An unscented sampling method, adapted from the unscented tranform used in the 
    state estimation community \cite{howell2020direct, julier2004unscented,
      lee2013sigma, manchester2016derivative}.
  \item A derivative-penalization method, which uses efficient mixed-mode differentiation
  to compute derivative information of parameter deviations with respect to the 
  quantum state trajectory.
\end{enumerate}
We apply these techniques to the fluxonium qubit presented in \cite{zhang2020universal}.
We also show that QOC can solve important problems for fluxonium-based qubits,
in particular taking advantage of the $T_{1}$-dependence of the controls
to mitigate longitudinal relaxation and
performing phase gates in arbitrary times.
To mitigate errors due to longitudinal
relaxation, we perform time-optimal control and
utilize an efficient optimization objective that does
not pay the increased computational cost of integrating a master equation.
Leveraging recent advances in trajectory optimization within the field of robotics, we
solve these optimization problems using the ALTRO solver, which uses iterative LQR 
(iLQR)---a differential-dynamic programming (DDP)-based indirect method similar to shooting
methods such as GOAT \cite{machnes2015gradient}, GRAPE
\cite{khaneja2005optimal, leung2017speedup}, and Krotov's \cite{goerz2019krotov}---within 
an augmented Lagrangian framework to handle nonlinear equality and inequality constraints at 
each time step \cite{howell2019altro}. 

This paper is organized as follows.
First, we introduce the ALTRO method in the context of QOC
in Section \ref{sec:background}.
We describe realistic constraints for the fluxonium and
map them to the ALTRO method in Section \ref{sec:fluxonium}. Then, we
outline a method for making the optimization aware of longitudinal
relaxation in Section \ref{sec:longitude}. Next, we outline three methods for achieving
robustness to static parameter deviations in Section \ref{sec:static}. Finally,
we employ the robust control techniques to mitigate 1/$f$ flux noise
in Section \ref{sec:stochastic}.
