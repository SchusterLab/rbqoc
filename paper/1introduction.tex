\section{Introduction}
Quantum optimal control (QOC) techniques are a class of optimization
algorithms for accurately and efficiently manipulating quantum systems.
Early techniques were proposed for nuclear magnetic resonance experiments
\cite{vandersypen2005nmr, kehlet2004improving, khaneja2005optimal,
  maximov2008optimal, nielsen2010optimal, skinner2003application, tosner2009optimal},
and applications now include superconducting circuits \cite{abdelhafez2020universal,
  chakram2020multimode, fisher2010optimal, gokhale2019partial,
  huang2020engineering, leng2019robust, leung2017speedup, li2020fast,
  xu2020nonadiabatic},
neutral atoms and ions \cite{brouzos2015quantum,
  de2008optimal, goerz2011quantum, guo2019high, jensen2019time,
  larrouy2020fast, omran2019generation, rosi2013fast, sorensen2019qengine,
  treutlein2006microwave, van2016optimal},
nitrogen-vacancy centers in diamond \cite{chou2015optimal,
  dolde2014high, geng2016experimental,
  nobauer2015smooth, poggiali2018optimal, rembold2020introduction, tian2019optimal},
and Bose-Einstien condensates \cite{amri2019optimal, sorensen2018quantum}.
For quantum computation,
optimal control techniques are employed to achieve high fidelity gates
while adhering to experimental constraints.
Experimental errors may cause the system to deviate
from the model used in optimization, leading
to poor experimental performance.
Robust control techniques improve upon
standard optimal control techniques by encoding
model uncertainties
in optimization objectives, yielding performance
guarantees over a range of parameters \cite{Zhou97,Morimoto00,Manchester18}.
We adapt robust control techniques from the robotics community to mitigate
realistic model uncertainties for
a superconducting fluxonium qubit.

QOC techniques have had tremendous success in
engineering high fidelity quantum gates \cite{
  chou2015optimal, dolde2014high, egger2013optimized, egger2014adaptive,
  grace2007optimal, heeres2017implementing, huang2014optimal,
  kelly2014optimal, leng2019robust, liebermann2016optimal,
  nebendahl2009optimal, rebentrost2009optimal, rebentrost2009optimal2,
  spiteri2018quantum, sporl2007optimal}. 
While standard QOC techniques can predict system behavior
with high accuracy, they are sensitive to experimental
errors such as parameter drift, noise, finite control 
resolution, and decoherence. Multiple techniques have been developed to address
these shortcomings. Analytic techniques to mitigate parameter deviation errors
include dynamic and geometric phase considerations
\cite{han2020experimental, xu2020nonadiabatic} and
composite pulses \cite{cummins2000use, cummins2003tackling,
  kupce1995stretched, merrill2014progress}.
To mitigate decoherence, Floquet techniques have been employed
\cite{huang2020engineering, mundada2020floquet}.
Numerical techniques to mitigate parameter deviation errors
include closed-loop methods \cite{egger2014adaptive, feng2018gradient, huang2020engineering,
  wittler2020integrated} and open-loop methods \cite{
  allen2019robust, carvalho2020error, reinhold2019controlling,
  rembold2020introduction, kosut2013robust}.
Numerical techniques to mitigate decoherence include
modeling master equations \cite{rembold2020introduction} and employing
Monte Carlo style quantum trajectories \cite{abdelhafez2019gradient}.

In this work, we study three robust control techniques that
make the system's state trajectory less sensitive
to static and time-dependent parameter deviations:
\begin{enumerate}
  \item A sampling method, similar to the work of \cite{allen2019robust,
    carvalho2020error, reinhold2019controlling, rembold2020introduction}.
  \item An unscented sampling method adapted from the unscented transform used in the 
    state estimation community \cite{howell2020direct, julier2004unscented,
      lee2013sigma, manchester2016derivative}.
  \item A derivative-penalization method, which uses efficient mixed-mode differentiation
    to compute derivative information of parameter deviations with respect to the 
    quantum state trajectory.
\end{enumerate}
We apply these techniques to the fluxonium qubit presented in \cite{zhang2020universal}.
We also show that QOC can solve important problems for fluxonium-based qubits,
in particular, taking advantage of the $T_{1}$-dependence of the controls
to mitigate longitudinal relaxation type decoherence and
performing phase gates in arbitrary times.
To mitigate errors due to longitudinal
relaxation, we perform time-optimal control and
utilize an efficient optimization objective that does
not pay the increased computational cost of integrating a master equation.
Leveraging recent advances in trajectory optimization within the field of robotics, we
solve these optimization problems using the ALTRO solver, which uses iterative LQR 
(iLQR)---a differential-dynamic programming (DDP)-based indirect method similar to shooting
methods such as GOAT \cite{machnes2015gradient}, GRAPE
\cite{khaneja2005optimal, leung2017speedup}, and Krotov's \cite{goerz2019krotov}---within 
an augmented Lagrangian framework to handle nonlinear equality and inequality constraints at 
each time step \cite{howell2019altro}. 

This paper is organized as follows.
First, we introduce the ALTRO method in the context of QOC
in Section \ref{sec:background}.
We describe realistic constraints for the fluxonium and
map them to the ALTRO method in Section \ref{sec:fluxonium}. Then, we
outline a method for making the optimization aware of longitudinal
relaxation in Section \ref{sec:longitude}. Next, we outline three methods for achieving
robustness to static parameter deviations in Section \ref{sec:static}. Finally,
we employ the robust control techniques to mitigate 1/$f$ flux noise
in Section \ref{sec:stochastic}.
