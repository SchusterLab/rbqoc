\section{Introduction}
Quantum optimal control (QOC) techniques are a class of optimization
algorithms for accurately and efficiently manipulating quantum systems.
Early techniques were proposed for nuclear magnetic resonance experiments
\cite{khaneja2005optimal}, and applications now include superconducting
circuits \cite{heeres2017implementing,
  huang2020engineering, leng2019robust, leung2017speedup, xu2020nonadiabatic},
neutral and ionized atoms \cite{van2016optimal}, nitrogen-vacancy centers in
diamond \cite{rembold2020introduction}, and Bose-Einstien condensates
\cite{sorensen2018quantum}. For quantum computation
QOC techniques are employed to achieve high fidelity gates
while adhering to experimental constraints.
The decision variables of the optimization problem are the time-dependent control
parameters, particular to the quantum system, that govern its evolution.
Experimental errors may cause the system evolution to deviate from that predicted in
optimization, hampering performance.
The field of classical control theory has developed robust control techniques
to encode experimental errors in optimization objectives, improving
experimental performance (citations appreciated).
In this work we employ robust control techniques to mitigate
realistic system parameter deviations and decoherence that arise when controlling
a superconducting fluxonium qubit.

Analytic and numerical techniques for QOC have been remarkably successful in
designing high fidelity gates (citations needed). In addition to the high
simulated fidelities which can be obtained with these techniques, it is desirable for the
resultant quantum state trajectory to be insensitive to experimental errors
such as parameter drift, noise, and finite control resolution. Quantum state
trajectories that are insensitive to these effects are said to be robust.
In line with this objective, recent work has sought to mitigate decoherence
and errors due to parameter deviations.
Considering the dynamical and geometric phases of quantum state
trajectories has led to analytic methods for mitigating
errors due to parameter deviations and pure dephasing
\cite{han2020experimental, merrill2014progress, xu2020nonadiabatic, zhang2020universal}.
Floquet techniques have been experimentally demonstrated to simultaneously mitigate
decoherence due to longitudinal relaxation and pure dephasing
\cite{huang2020engineering, mundada2020floquet}.
Numerical QOC techniques have been adapted to mitigate longitudinal relaxation
by modeling master equations \cite{rembold2020introduction} and employing
Monte Carlo style quantum trajectories \cite{abdelhafez2019gradient}.
Additionally, efforts have been made to incorporate experimental feedback
into optimization \cite{huang2020engineering}. Comprehensive tools that interleave characterization
and pulse design are under active development \cite{wittler2020integrated}.

In this work we develop robust trajectory optimization techniques to mitigate
errors arising from longitudinal relaxation, parameter deviations, and 1/$f$ flux noise
on the fluxonium qubit presented in \cite{zhang2020universal}. We perform these optimization using the ALTRO
method \cite{howell2019altro} which combines the iterative, indirect shooting
flavor of methods such as GOAT \cite{machnes2015gradient}, GRAPE
\cite{khaneja2005optimal, leung2017speedup}, and Krotov's \cite{goerz2019krotov}
with the augmented Lagrangian method. The augmented Lagrangian method
allows us to enforce arbitrary, simultaneous constraints and achieve
fast convergence without being restricted to the constraint
manifold. To mitigate errors due to longitudinal
relaxation, we perform time-optimal control and
encode the dependence of longitudinal relaxation
on the controls in an efficient optimization objective that does
not pay the increased computational cost of integrating a master equation.
Additionally, we consider three robust
robust trajectory opitmization techniques to make the
quantum state trajectory robust to parameter deviations.
The first is the sampling method, which has been applied
previously in the context of QOC
%%TODO: the citations for unscented sampling
%% and sampling are the same as in section V
\cite{carvalho2020error, reinhold2019controlling, rembold2020introduction}.
The second is the unscented sampling
method, which derives from the unscented transformation used
for non-linear Kalman filtering
\cite{howell2020direct, julier2004unscented, lee2013sigma, manchester2016derivative}.
We propose a third method, which uses derivative information
of parameter deviations with respect to the quantum state trajectory.
We make this method efficient by employing mixed-mode differentiation.

This paper is organized as follows.
First, we introduce the ALTRO method in the context of QOC.
We describe realistic constraints for the fluxonium and
map them to the ALTRO method. Then, we
outline a method for making the optimization aware of longitudinal
relaxation. Next, we outline three methods for achieving
robustness to static parameter deviations. Finally,
we employ the robust control techniques to mitigate 1/$f$ flux noise.
