\section{Introduction}
Quantum optimal control (QOC) is a class of optimization
algorithms for accurately and efficiently manipulating quantum systems.
Early techniques were proposed for nuclear magnetic resonance experiments
\cite{vandersypen2005nmr, kehlet2004improving, khaneja2005optimal,
  maximov2008optimal, nielsen2010optimal, skinner2003application, tosner2009optimal},
and applications now include superconducting circuits \cite{abdelhafez2020universal,
  chakram2020multimode, egger2013optimized, fisher2010optimal, gokhale2019partial,
  huang2014optimal, heeres2017implementing, kelly2014optimal, leng2019robust,
  leung2017speedup, li2020fast,
  liebermann2016optimal, reinhold2019controlling,
  rebentrost2009optimal, rebentrost2009optimal2, spiteri2018quantum,
  sporl2007optimal},
neutral atoms and ions \cite{brouzos2015quantum,
  de2008optimal, grace2007optimal, goerz2011quantum, guo2019high, jensen2019time,
  larrouy2020fast, nebendahl2009optimal, omran2019generation,
  rosi2013fast,
  treutlein2006microwave, van2016optimal},
nitrogen-vacancy centers in diamond \cite{chou2015optimal,
  dolde2014high, geng2016experimental,
  nobauer2015smooth, poggiali2018optimal, rembold2020introduction, tian2019optimal},
and Bose-Einstein condensates \cite{amri2019optimal, doria2011optimal,
  sorensen2019qengine, sorensen2018quantum}.
In the context of quantum computation,
optimal control is employed to achieve high-fidelity gates
while adhering to experimental constraints.
Experimental errors such as parameter drift, noise, and
finite control resolution cause the system to deviate
from the model used in optimization, hampering
experimental performance
\cite{chakram2020multimode, heeres2017implementing, klimov2020snake,
  omran2019generation, reinhold2019controlling}.
Robust control improves upon
standard optimal control by encoding
model parameter uncertainties
in optimization objectives, yielding performance
guarantees over a range of parameter values \cite{Zhou97,Morimoto00,Manchester18}.
We adapt robust control techniques from the robotics community to mitigate
parameter-uncertainty errors for
a superconducting fluxonium qubit.

Analytically-derived control pulses that mitigate parameter-uncertainty
errors include composite pulses \cite{cummins2000use, cummins2003tackling,
  kupce1995stretched, merrill2014progress},
pulses designed by considering dynamic and geometric phases
\cite{han2020experimental, xu2020nonadiabatic}, and
pulses obtained with the DRAG scheme \cite{motzoi2009simple}.
As compared to analytical techniques, QOC is advantageous for
designing pulses that consider all experimental constraints and
performance tradeoffs
\cite{leung2017speedup},
and for constructing operations without a known analytic solution
\cite{chakram2020multimode, heeres2017implementing}.
Accordingly, recent work has sought to achieve robustness in QOC
frameworks using closed-loop methods \cite{egger2014adaptive, feng2018gradient,
  li2017hybrid, wittler2020integrated} and open-loop methods \cite{
  allen2019robust, carvalho2020error, reinhold2019controlling,
  rembold2020introduction, kosut2013robust, niu2019universal}.

In this work, we study three robust control techniques that
make the system's quantum state trajectory less sensitive to uncertainties of static and time-dependent parameters:
\begin{enumerate}
\item A sampling method, similar to the work in Refs.~\cite{allen2019robust,
  ball2020software, carvalho2020error, khaneja2005optimal,
  reinhold2019controlling, rembold2020introduction}.
\item An unscented sampling method \cite{howell2020direct, lee2013sigma, thangavel2020robust}
  adapted from the unscented transform \cite{julier2004unscented,
    uhlmann1995dynamic} used in state estimation.
  \item A derivative method, which penalizes the sensitivity of the quantum state trajectory
    to uncertain parameters.
\end{enumerate}
We apply these techniques to the fluxonium qubit presented in \cite{zhang2020universal}.
We also show that QOC can solve important problems associated with
fluxonium-based qubits: exploiting the $T_{1}$-dependence of the controls \textcolor{blue}{[as in the microwave generator output is $T_1$ dependent?]} 
to mitigate depolarization
and synchronizing the phase of qubits with distinct frequencies.
To mitigate \textcolor{blue}{[close-range repetition]} depolarization,
we perform time-optimal control and
employ an efficient depolarization model
for which the computational cost is independent of the
Hilbert space dimension.
Leveraging recent advances in trajectory optimization within the field of robotics, we
solve these optimization problems using ALTRO (Augmented Lagrangian TRajectory Optimizer)
\cite{howell2019altro}, which can enforce constraints on
the control parameters and the quantum state trajectory.

This paper is organized as follows.
First, we describe ALTRO in the context of QOC
in Section \ref{sec:background}.
We outline realistic constraints for operating the fluxonium and
define the associated QOC problem in Section \ref{sec:fluxonium}.
Then, we formulate a method for suppressing depolarization
in Section \ref{sec:longitude}. Next, we describe three techniques for achieving
robustness to static parameter uncertainties in Section \ref{sec:static}. We
adapt the same techniques to mitigate 1/$f$ flux noise
in Section \ref{sec:stochastic}.
