\section{Longitudinal Relaxation}
The probability of longitudinal relaxation metric and gate error due to longitudinal
relaxation metric are compared in Table I for the experiment shown in Figure 1c.
The relative performance between the analytic and numerical techniques we study
is similar across the two metrics.

\begin{table}[h!]
  \begin{tabular}{c | c | c}
    Gate & $P_{1\textrm{A}} / P_{1\textrm{N}}$ & GE$_{\textrm{A}}$ / GE$_{\textrm{N}}$\\
    \hline
    Z/2 & 5.745/1.149 = 5.000 & 1.776/0.371 = 4.787\\
    Y/2 & 5.253/1.157 = 4.540 & 1.539/0.370 = 4.159\\
    X/2 & 16.251/2.660 = 6.109 & 5.347/0.863 = 6.196\\
  \end{tabular}
  \caption{
    Single gate probability of longitudinal relaxation
    ratios and single gate error due to longitudinal relaxation
    ratios for the analytic and numerical methods. Both probabilities
    and gate errors are shown in units of $10^{-5}$.
  }
\end{table}

To compute the gate error due to longitudinal relaxation,
we require the final state of the quantum system subject
to longitudinal relaxation. To obtain the final state we employ
the Lindblad master equation. This equation takes the form
\begin{equation}
  \frac{d}{dt} \rho = \frac{-i}{\hbar} [H, \rho]
  + \sum_{i = 1}^{n^{2} - 1} \gamma_{i} (L_{i} \rho L_{i}^{\dagger}
  - \frac{1}{2} \{L_{i}^{\dagger} L_{i}, \rho\})
\end{equation}
Here $\rho = \ket{\psi}\bra{\psi}$ is the density matrix, $n = \textrm{dim}(\mathcal{H})$,
and $[\cdot, \cdot], \{\cdot, \cdot \}$ are the algebraic commutator and anti-commutator.
For longitudinal relaxation $\gamma_{\uparrow} = T_{1, \uparrow}^{-1}$,
$\gamma_{\downarrow} = T_{1, \downarrow}^{-1}$,
$L_{\uparrow} = \sigma^{+}/2$, and
$L_{\downarrow} = \sigma^{-}/2$ where $\sigma^{\pm} = \sigma_{x} \pm i \sigma_{y}$.
Both $T_{1, \uparrow}$ and $T_{1, \downarrow}$ are obtained from the spline shown in Figure 1b.
The $T_{1}$ values in this spline are obtained experimentally by driving the qubit at the desired flux amplitude
and monitoring the resultant decay. For more details consult \cite{zhang2020universal}.

Exponential integrators can be employed to integrate the Lindblad master equation
using the Vectorization/Choi-Jamiolkowski isomorphism \cite{Landi2018}
\begin{equation}
  \frac{d}{dt} \textrm{vec}({\rho}) = \hat{\mathcal{L}} \textrm{vec}({\rho})
\end{equation}
\begin{equation}
  \begin{aligned}
    \hat{\mathcal{L}} &= -i(I \otimes H - H^{T} \otimes I)\\
    &+ \sum_{i = 1}^{n^{2} - 1} \gamma_{i}
    (L_{i}^{*} \otimes L_{i} - \frac{1}{2} (I \otimes L_{i}^{\dagger}L_{i}
    - L_{i}^{T}L_{i}^{*} \otimes I))
  \end{aligned}
\end{equation}
Here $\rho = \sum_{i, j} \alpha_{i, j} \ket{i}\bra{j}$
and $\textrm{vec}(\rho) = \sum_{i, j} \alpha_{i, j} \ket{i} \otimes \ket{j}$.
We use zero-order hold on the controls so the integration is exact
$\textrm{vec}(\rho_{k + 1}) = \exp(\Delta t_{k} \hat{\mathcal{L}}_{k}) \textrm{vec}(\rho_{k})$.
This isomorphism transforms $(n^{2} \times n^{2}) \times (n^{2} \times n^{2})$
matrix-matrix multiplications to $(n^{4} \times n^{4}) \times n^{4}$ matrix-vector
multiplications. For small $n$ and zero-order held controls, we find that it is
faster to use an exponential integrator on the vectorized equation than to perform
Runge-Kutta on the unvectorized equation. The latter requires decreasing the integration
time step to maintain accuracy, resulting in more knot points.

