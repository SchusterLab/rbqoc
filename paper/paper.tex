\documentclass[
  amsfonts,
  amsmath,
  amssymb,
  aps,
  nobibnotes,
  prl,
  twocolumn,
]{revtex4-2}

% imports
\usepackage{appendix}[titletoc]
\usepackage[english]{babel}
\usepackage[justification=centering]{caption}
\usepackage{float}
\usepackage{graphicx}
\usepackage{hyperref}
\usepackage[utf8]{inputenc}
\usepackage{layouts}
\usepackage{optidef}
\usepackage{physics}
\usepackage{setspace}
\usepackage[labelformat=simple]{subcaption}
\usepackage{xcolor}

% configure imports
\DeclareMathOperator*{\argmin}{arg\,min}
\newcommand{\todo}[1]{\textcolor{red}{TODO: #1}}
\definecolor{darkgreen}{RGB}{46, 184, 46}
\newcommand{\half}{\frac{1}{2}}
\newcommand{\R}{\mathbb{R}}
\captionsetup[subfigure]{labelformat=empty, skip=0pt}
\restylefloat{table}
\renewcommand\thesubfigure{(\alph{subfigure})}



\begin{document}

%% TITLE
\title{Robust Quantum Optimal Control}

\author{Thomas Propson}
\email{tcpropson@uchicago.edu}
\affiliation{
  Department of Physics, University of Chicago, Chicago, Illinois 60637, USA
}

\date{\today}

\maketitle


%% ABSTRACT
\begin{abstract}
  This is a paper about robust quantum optimal control.
\end{abstract}


%% SECTION 1
\section{Introduction}
[Existing work] The leading model of universal quantum computation is gate-based. There are analytic
techniques to construct gates \cite{huang2020engineering}
\cite{merrill2014progress} \cite{han2020experimental} \cite{wang2007nonadiabatic}.
Most methods focus on optimizing a few aspects of the gate, but not all
experimentally relevant aspects. There is a growing literature on numerical
techniques to construct gates \cite{leung2017speedup} \cite{abdelhafez2019gradient}.
%% TODO: Cite CRAB, GOAT, KROTOV, etc.
These methods formulate the quantum optimal control problem as unconstrained.
They typically rely on zeroth-order or first-order optimizers. They are not
sofisticated enough to handle all of the relevant constraints simultaneously.
Although some analytic techniques exist to design pulses robust
to decoherence, no numerical techniques have been presented to
design pulses robust to decoherence (as far as the author is aware).

[This work] We employ the trajectory optimization literature to formulate the quantum
optimal control problem as a constrained optimization problem.
We study the quantum optimal control problem on the fluxonium.
We outline experimentally realistic constraints and map them to
the trajectory optimization framework. For the device we study
we achieve a 2x increase in $T_{1}$ times.
We present two methods for achieving robustness to system parameter deviations,
and compare to existing dynamic decoupling methods. We find that 
our methods beat dynamic decoupling and mitigate dephasing by order X.


[Outline]  First we formulate the quantum optimal control problem
in the trajectory optimization framework. Then, we introduce the dynamics
of the fluxonium device and outline experimental considerations
relevant to gate construction. Next we outline a method for
making the optimization $T_{1}$ aware. Finally, we present
some methods for engineering robustness to decoherence and
compare them to existing techniques.


%% SECTION 2
\section{QOC + AL-iLQR}
[QOC Problem Statement] I have some initial
configuration and I want to reach some target configuration,
either single- or multi-state transfer, subject to the dynamics
\begin{align}
  \frac{\partial \ket{\psi}(t)}{\partial t} &= -/frac{i}{\hbar} H(t) \ket{\psi}(t)
\end{align}
Common numerical techniques include approximating the analytic unitary propagator solution,
or employing explicit/implicit Runge-Kutta methods of the form
\begin{align}
  \ket{\psi_{k + 1}} &= \ket{\psi_{k}} + \frac{\partial \ket{\psi_{k}}}{\partial t} \cdot dt
\end{align}
The interesting part is that your hamiltonian $H(t)$ has a time-dependent
control parameter $u(t)$ that the experimentalist gets to control, e.g. flux threading
a superconducting junction.
\begin{align}
  H(t) &= H_{0} + \sum_{i = 1}^{m} a_{m}(u_{m}(t)) H_{m}\\
\end{align}
Most often $a_{m}(u) = u$ but any arbitrary dependence is allowed.
The technique of adjusting $\ket{\psi_{N}}$ based on $\textbf{u}(t)$
is the goal of the quantum optimal control optimizer. This domain is
know as sensitivity analysis. GRAPE does
\begin{align}
  \min_{U:N-1 \times m} &1 - {\lvert \braket{\psi_{N}(U)}{\psi_{t}} \lvert}^{2}\\
\end{align}
with
\begin{align}
  U^{(k + 1)} &= U^{(k)} - \alpha \textbf{\nabla_{\textbf{U}}} (1 - {\lvert \braket{\psi_{N}(U)}{\psi_{t}} \lvert}^{2})\\
\end{align}
In general they add more cost functions than target fidelity.
%% TOOD: Other update procedures might differ, e.g. Krotov does some sequential updates
%% and Houck, Google use black box optimizers.

[AL-iLQR Problem Statement] Trajectory optimization gives us gaurantees
about our updates via Ricatti recursion
and allows us to put constraints on our cost functions.
\begin{align}
  %% This is e.q. 2 from ALTRO paper
  \min_{U:N-1 \times m} &\mathcal{L}_{N}(X_{N}, \lambda_{N}, \mu_{N}) + \sum_{k = 1}^{N - 1} \mathcal{L}_{k}(X_{k}, U_{k}, \lambda_{k}, \mu_{k})\\
  \mathcal{L}_{k}(X_{k}, U_{k}, \lambda_{k}, \mu_{k}) &= (\lambda_{k} + \frac{1}{2}I_{\mu_{k}} c_{k}(X_{k}, U_{k}))^{T} c_{k}(X_{K}, U_{K})
  &+ (X_{k} - X_{f})^{T} Q (X_{k} - X_{f}) + U^{T} R U\\
\end{align}
The important point is that there is an update step (e.q. 17 from ALTRO paper) where we send \lambda -> \infty
and get all of the nice convergence properties. The weights are adjusted dynamically between iterations until
all of our constraints are satisfied. The Markovian decision structure of the problem allows
us to apply differentiable dynamic programming to garauntee that the update for each control $U_{k}$ is
opimal, as apposed to the greedy updates of first-order optimizers like the naive gradient descent.

%% TODO: (?) Figure comparing update procedures


%% SECTION 3
\section{Our Problem}
[Fluxonium + Device Characterization] In the two-level
approximation we have
\begin{align}
  H/h &= \omega_{q} \frac{\sigma_{z}}{2} + A(\Phi_{ext}) \frac{\sigma_{x}}{2}\\
\end{align}
This approximation is good up to the avoided crossing at 0.35 $\Phi_{0}$. We get $A$
by converting via $\bra{g} \hat{\phi} \ket{e}$.
%% TODO: list device parameters

[Constraints] We want constraints on our pulses. We want pulses start and end at zero
for concatenation. We want pulses to have zero net flux to mitigate
hysteresis in flux bias lines.
%% TODO: See ref. 28 in Helin paper
We want the amplitude to be constrained
$\delta \Phi_{\textrm{ext}} \sim 0.06 \Phi_{0}$
so the two-level approximation stays valid.
We want our state to obey normalization conditions,
mitigating numerical error in simulaiton.

We test on the basis gates $X/2, Y/2, Z/2$. Universal
up to arbitrary Z rotation.


%% SECTION 4
\section{Robustness to $T_{1}$-type Noise}



%% SECTION 5
\section{Robustness to $T_{\phi}$-type Noise}



\bibliography{refs}

\end{document}
