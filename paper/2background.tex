\section{Background \label{sec:background}}
In this section, we
review the QOC problem statement
and reintroduce the ALTRO solver \cite{howell2019altro}.
QOC concerns a vector of time-dependent control parameters $u(t)$
that steer the evolution of a (quantum) state $\ket{\psi(t)}$.
The evolution of the state is governed by the time-dependent
Schr{\"o}dinger equation (TDSE),
\begin{equation}
  i \hbar \frac{d}{dt} \ket{\psi(t)} = H(u(t), t) \ket{\psi(t)}
  \label{eq:tdse}
\end{equation}
The Hamiltonian $H(u(t), t)$ is determined by the quantum system.
The QOC problem is to find the
controls that minimize a functional $J(u(t))$.
To make the problem numerically tractable,
the controls and state are discretized into $N$ knot points (time steps).
In the case of a single state transfer problem, the functional is
the infidelity of the desired final state and the initial state evolved
to the final knot point,
$J(u) = 1 - {\lvert \braket{\psi_{f}}{\psi_{N}(u)} \rvert}^{2}$.
In general, $J(u)$ is a linear combination of cost functions on the state
as well as cost functions on the controls \cite{leung2017speedup}.
Standard QOC solvers compute derivatives of the functional $\nabla_{u} J(u)$,
which are used to implement first-order optimization methods
\cite{machnes2015tunable, khaneja2005optimal, leung2017speedup, goerz2019krotov}.


\myrem{Alternatively, the QOC problem can be formulated as a trajectory optimization problem 
and solved using a variety of specialized solvers developed by the robotics community
\cite{Schulman13,Tedrake16,Hereid2017FROST,howell2019altro}.
Trajectory optimization problems are typically of the following form: 
\begin{mini!}[2]
    {x_{1:N},u_{1:N-1}}{\ell_f(x_N) + \sum_{k=1}^{N-1} \ell(x_k,u_k)}{}{} \label{eq:gcostfun}
    \addConstraint{x_{k+1} = f(x_k,u_k)} \label{eq:gdyncon}
    \addConstraint{g_k(x_k,u_k)}{\leq 0} \label{eq:ineqcon}
    \addConstraint{h_k(x_k,u_k)}{=0} \label{eq:eqcon}
\end{mini!}
where $\ell_f$ and $\ell$ are the final and stage cost functions, $x_k \in \R^l$ and
$u_k \in \R^m$ are the state and control variables,
$f(x_k,u_k)$ is the discrete dynamics function, and $g_k(x_k,u_k)$ and
$h_k(x_k,u_k)$ are the inequality
and equality constraints, potentially including initial and final conditions,
at time step $k$.}

\myadd{Alternatively, the QOC problem can be formulated as a trajectory optimization problem 
and solved using a variety of specialized solvers developed by the robotics community
\cite{Schulman13, Tedrake16, Hereid2017FROST, howell2019altro}.
The functional $J(u)$ is divided into its constituent
cost functions at each knot point $\ell_{k}(x_{k}, u_{k})$,
where $k \in \{1, ..., N\}$ denotes the knot point,
$x_{k}$ is the augmented state vector,
and $u_{k}$ is the augmented control vector.
The augmented state includes all variables of interest that are not decision variables.
In the case of QOC, the state is a subset of the
augmented state $\ket{\psi_{k}} \subseteq x_{k}$.
The solver we use does not currently support complex numbers, so we employ the isomorphism
$\mathcal{H}(\mathbb{C}^{n}) \cong \mathcal{H}(\mathbb{R}^{2n})$
given in \cite{leung2017speedup},
\begin{equation}
  H \ket{\psi} \cong \begin{bmatrix} H_{\textrm{re}} & -H_{\textrm{im}}
    \\ H_{\textrm{im}} & H_{\textrm{re}}\end{bmatrix}
  \begin{bmatrix} \ket{\psi}_{\textrm{re}} \\ \ket{\psi}_{\textrm{im}}\end{bmatrix}
  \label{eq:isomorphism}
\end{equation}
The augmented state trajectory obeys the physics of the system if
the dynamics constraint is satisfied
$x_{k + 1} = f(x_{k}, u_{k})$. For QOC, the discrete
dynamics function $f(x_{k}, u_{k})$ propagates the state by
integrating the TDSE \ref{eq:tdse} with Runge-Kutta methods \cite{jorgensen2011numerical}
or exponential integrators \cite{auer2018magnus, berland2005solving, einkemmer2017performance}.}

\myadd{Additional constraints on the augmented states and augmented controls
are encoded in constraint functions.
The constraint functions are put into a form such that, when the constraint is satisfied,
inequality constraint functions obey $g_{k}(x_{k}, u_{k}) \leq 0$ and
equality constraint functions obey $h_{k}(x_{k}, u_{k}) = 0$. The constraint
functions may be vector-valued to encode multiple constraints of each type.
If a constraint function takes a positive value, the
value is called the constraint's violation and
the constraint is said to be active.
The trajectory optimization problem can be stated concisely,
\begin{mini!}[2]
    {x_{1:N},u_{1:N-1}}{\ell_{N}(x_N) + \sum_{k=1}^{N-1} \ell_{k}(x_k,u_k)}{}{} \label{eq:gcostfun}
    \addConstraint{x_{k+1} = f(x_k,u_k)} \label{eq:gdyncon}
    \addConstraint{g_k(x_k,u_k)}{\leq 0} \label{eq:ineqcon}
    \addConstraint{h_k(x_k,u_k)}{=0} \label{eq:eqcon}
\end{mini!}}


\myrem{Many techniques have been proposed for solving
\eqref{eq:gcostfun}--\eqref{eq:eqcon}. Standard 
methods include direction collocation \cite{Hargraves87} and differential-dynamic programming
(DDP) \cite{Mayne1966a}. Recent state-of-the-art solvers, such as ALTRO \cite{howell2019altro},
have combined principles from both of these approaches.}

\myadd{Many techniques have been proposed for solving
\eqref{eq:gcostfun}-\eqref{eq:eqcon}. Standard 
methods include direction collocation \cite{Hargraves87} and differential-dynamic programming
(DDP) \cite{Mayne1966a}. Recent state-of-the-art solvers, such as ALTRO,
have combined principles from both of these approaches. ALTRO uses iLQR
\cite{Li2004a} as the internal solver of an augmented Lagrangian method (ALM)
\cite{lantoine2012hybrid, plancher2017constrained}
and employs a projected Newton method \cite{bertsekas1982projected, rao1998application}
in its final solving stage.}

\myrem{ALTRO uses iterative LQR (iLQR) \cite{Li2004a} as the internal solver of an augmented 
Lagrangian method (ALM). iLQR solves an unconstrained trajectory optimization problem 
using a backward Riccati recursion to derive a closed-loop linear feedback law about the current 
trajectory. By simulating the system forward with the feedback law, the trajectory is 
brought closer to the (locally) optimal trajectory. DDP-based solvers such as iLQR
are popular since they are very computationally efficient, are always dynamically feasible,
and provide a locally stabilizing closed-loop control policy about the optimal trajectory. However,
standard implementations have no ability to deal with nonlinear equality and inequality 
constraints. ALM handles constraints by successively solving unconstrained minimization 
problems of the form:
\begin{mini}[2]
    {z}{f(z) + \lambda^T c(z) + \half c(z)^T I_\mu c(z)}{}{}
    \label{opt:alm}
\end{mini}
where $f(z)$ is the objective function, $c(z) : \R^{l + m} \mapsto \R^p$ is the constraint 
function, $\lambda \in \R^p$ is a Lagrange multiplier estimate, and $I_\mu$ is a diagonal matrix
of penalty weights, $\mu$, whose magnitudes depend on whether
the corresponding constraint is active or inactive.
For ALTRO, $f(z)$ is \eqref{eq:gcostfun}, $c(z)$ is the 
concatenation of $g_k$ and $h_k$, and $z$ is the concatenation of the states and controls 
across all time steps. After minimizing \eqref{opt:alm} using iLQR,
the Lagrange multiplier estimates are updated according to,
\begin{equation}
	\lambda \gets \lambda + \mu c(z)
\end{equation}
the penalty terms are updated, and the process repeats until convergence.}

\myadd{iLQR simulates the augmented state forward
  using the discrete dynamics function 
  and computes derivatives of the cost functions
  to update the augmented controls. This procedure is known as
  indirect shooting, and iLQR shares this approach with GOAT \cite{machnes2015tunable},
  GRAPE \cite{khaneja2005optimal, leung2017speedup}, and Krotov's method \cite{goerz2019krotov}.
  iLQR differs from these solvers
  in its update procedure. iLQR computes the locally optimal update for the augmented
  controls using a dynamic programming algorithm, and a line search is performed
  in the direction of this update to ensure a decrease in the sum of the cost functions.}

\myadd{Indirect shooting solvers such as iLQR are popular
for solving unconstrained trajectory optimization problems \eqref{eq:gcostfun}-\eqref{eq:gdyncon},
but standard implementations have no ability to handle nonlinear equality and inequality
constraints \eqref{eq:ineqcon}-\eqref{eq:eqcon}.
One way to handle constraints is to employ a projected gradient
method \cite{clarkson2010coresets, hauswirth2016projected}.
However, within the indirect shooting framework,
projected gradient methods can
only handle constraints on the augmented controls,
not the augmented states.
Another technique is to promote the constraint functions to cost functions \cite{leung2017speedup}.
A simple implementation of this approach does not garauntee that the constraints
are satisfied, as the solver trades
minimization of the cost functions and constraint functions against each other.
ALM promotes the constraint functions to cost functions, and adaptively
adjusts a Lagrange multiplier estimate for each constraint function to ensure
the constraints are satisfied. ALM considers a decision variable $z \in \R^{s}$ with $s$ given,
a cost function $f(z) : \R^{s} \mapsto \R$, and a constraint function
$c(z) : \R^{s} \mapsto \R^{p}$ encoding $p$ constraints.
For ALTRO, $z$ is the vector concatenation of the augmented states and augmented controls across
all knot points, $f(z)$ is the sum of the trajectory optimization
cost functions \eqref{eq:gcostfun},
and $c(z)$ is the vector concatenation of the trajectory
optimization constraint functions \eqref{eq:ineqcon}-\eqref{eq:eqcon}.
ALM successively solves unconstrained minimization 
problems of the form:
\begin{mini}[2]
    {z} {f(z) + \lambda^T c(z) + \half c(z)^T I_\mu c(z)}{}{}
    \label{opt:alm}
\end{mini}
$\lambda \in \R^p$ is a Lagrange multiplier estimate, and $I_\mu$ is a diagonal matrix
of penalty weights, $\mu \in \R^{p}$, whose magnitudes depend on whether
the corresponding constraint is active or inactive.
After minimizing \eqref{opt:alm} using iLQR,
the Lagrange multiplier estimates are updated according to,
\begin{equation}
  \lambda \gets \lambda + I_\mu c(z)
  \label{eq:almupdate}
\end{equation}
the penalty weight is updated, and the process repeats until convergence.}

\myrem{ALM converges superlinearly but tends to exhibit
  slow constraint convergence near the optimal 
  solution due to poor numerical conditioning. To address this shortcoming, ALTRO provides a 
  solution-polishing phase that takes 1-2 Newton steps on the active constraint set to provide 
  machine-precision constraint satisfaction. For more information on the details of the ALTRO
  solver, see \cite{howell2019altro, Jackson2020altroc}.}

\myadd{ALM converges superlinearly, but poor numerical conditioning may lead
to small decreases in the constraint violations near the locally optimal solution
\cite{bertsekas1996constrained}. To address this shortcoming, ALTRO
projects the solution from the ALM solving stage onto the constraint manifold using
a projected Newton method, achieving ultra-low
constraint violations $\sim 1 \cdot 10^{-8}$.
For more information on the details of the ALTRO
solver, see \cite{howell2019altro, Jackson2020altroc}.}

\myrem{As opposed to standard QOC solvers, the ALTRO solver provides a unified framework
for imposing arbitrary, simultaneous constraints
on both the states and the controls that converge superlinearly to tight tolerances. 
First-order solvers such as GRAPE
use unconstrained optimization to satisfy desiderata on the states,
such as achieving a desired gate fidelity, forbidding occupations of particular quantum states,
and smoothing the controls below a threshold. The unconstrained approach relies
on a careful choice of competing penalty weights, which is cumbersome and often intractable.
To impose constraints on the controls, such as amplitude restrictions, first-order solvers
employ projected gradient descent methods which restrict
optimization to the constraint manifold and
may hinder convergence to the optimal solution \cite{clarkson2010coresets,
  hauswirth2016projected}.
Conversely,
the ALTRO solver delays projecting onto the constraint manifold defined by
\eqref{eq:ineqcon} and \eqref{eq:eqcon} until the ALM
solve is already at a coarse tolerance, achieving machine-precision satisfaction
for constraints on the controls and the states. These advantages are critical
for this work, where multiple constraints must be satisfied in addition
to minimizing auxiliary metrics.}

\myadd{As opposed to standard QOC solvers, ALTRO
is able to satisfy constraints
on both the augmented states and the augmented controls to tight tolerances.
This advantage is crucial for this work, where multiple medium-priority cost functions
are minimized subject to many high-priority constraints.}
