\section{Robustness to Time-Dependent Parameter Uncertainty \label{sec:stochastic}}
An additional source of experimental error arises from time-dependent
Hamiltonian parameter uncertainty. For many flux-biased and inductively-coupled
superconducting circuit elements, magnetic flux noise is a significant
source of coherent errors \todo{coherent errors?}. Flux noise
modifies the fluxonium Hamiltonian \eqref{eq:hamiltonian}
by $a(t) \rightarrow a(t) + \delta a(t)$.
The spectral density of flux noise is known to
follow a 1/$f$ distribution
\cite{bialczak20071f, koch2007model,
  yoshihara2006decoherence, yoshihara2010correlated},
so the noise is dominated by low-frequency components.
The analytic gate considered here
takes advantage of the low-frequency characteristic and
treat the noise as quasi-static, performing a generalization of the spin-echo
technique to compensate for erroneous drift.

Additionally, we modify the robust control techniques presented
in the previous section to combat 1/$f$ flux noise.
The unscented sampling method is modified so that the sample states
are subject to 1/$f$ flux noise. The noise
is generated by filtering white noise sampled from a standard
normal distribution with a finite impulse response filter \cite{saspweb2011}.
The noise is then scaled by the 
flux noise amplitude of our device $A_{\Phi} = 5.21 \mu \Phi_{0} \implies
\sigma_{a} = 2.5 \cdot 10^{-5} \textrm{GHz}$.
In principle, we could modify the sampling method
similarly; however, we choose to sample statically $a(t) \rightarrow a(t) + \sigma_{a}$
for comparison. The derivative methods require no algorithmic modification
from the static case, but the TDSE is now differentiated with respect
to $a(t)$ instead of $f_{q}$ as in \eqref{eq:d1dyn}.

We compare the gate errors due to 1/$f$ flux noise for $X/2$
gates constructed with the robust control techniques
and the analytic method. To compute the gate error,
an initial state is evolved
under the fluxonium Hamiltonian \eqref{eq:hamiltonian}
where the optimized flux amplitude is modified $a(t) \rightarrow a(t) + \delta a(t)$.
The flux noise $\delta a(t)$ is generated as
we described for the unscented sampling method.
Each of the $1000$ averages we perform uses a different random seed for
generating the flux noise.
We simulate successive applications of the gate constructed by each method
and compute the cumulative gate error
after each application, see Figure \ref{fig:stochastic}.
Both the analytic
and numerical gates yield single gate errors
sufficient for quantum error correction.
Despite converging on qualitatively different solutions, the
numerical gates perform similarly in the concatenated
gate application comparison. Their gate errors
after $200$ gate applications $\sim 11 \mu\textrm{s}$ are
two orders of magnitude less than the gate error produced by the analytic gate.
\todo{motional narrowing}.
1/$f$ flux noise is a significiant source of coherent errors in NISQ applications and
these numerical techniques offer effective avenues to mitigate it.
