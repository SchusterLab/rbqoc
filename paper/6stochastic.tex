\section{Robustness to Stochastic Parameter Deviations \label{sec:stochastic}}
An additional source of experimental error arises from stochastic, time-varying
Hamiltonian parameter deviations. For many flux-biased and inductively-coupled
superconducting circuit elements, magnetic flux noise is a significant
source of coherent errors. Magnetic flux noise
modifies the fluxonium qubit's amplitude from its nominal value by an amount $\delta a$.
It is well studied that the spectral density of $\delta a$ follows a
1/$f$ distribution for a range of devices, consisting primarily of low frequency
noise (citations needed). Analytic methods to combat flux noise
take advantage of the low frequency characteristic and
treat the noise as quasi-static, performing generalizations of the spin-echo technique
to compensate for erroneous drift. This is the strategy employed by the analytic gate
considered here.

We compare the analytic gate and those produced by
the numerical methods discussed in the previous section
on the task of realizing a $X/2$ gate subject to 1/$f$ flux noise.
The flux noise is generated by
filtering white noise sampled from a standard normal distribution with a finite
impulse response filter \cite{saspweb2011}
\footnote{\url{https://ccrma.stanford.edu/~jos/sasp/Example_Synthesis_1_F_Noise.html}}.
It is then scaled by the 
flux noise amplitude of our device $A_{\Phi} = 5.21 \mu \Phi_{0} \implies
\delta a \sim 2.5 \cdot 10^{-5} \textrm{GHz}$.
The unscented sampling method is modified so that its sampled deviations
follow a 1/$f$ distribution by carrying the state of a finite impulse response filter
in the augmented state vector. In principle the basic sampling method could be modified
similarly but we choose to sample statically at $\delta a$ for comparision. The derivative
methods require no modification from the static case. \todo{can you provide a little more 
detail on how this noise factors into the dynamics? Are you just adding some white 
noise to $a(t)$? Are you also considering uncertainty in $f_q$ at the same time?}

We simulate successive applications of the gate constructed by each method
and compute the cumulative gate error
after each application, see Figure \ref{fig:stochastic}. Both the analytic
and numerical methods achieve single gate errors
sufficient for quantum error correction. Despite converging on qualitatively different solutions, the
numerical methods perform similarly in the concatenated gate application comparision. They achieve a two
order of magnitude cumulative gate error reduction over the analytic method after $200$
gate applications $\sim 11 \mu\textrm{s}$.
1/$f$ flux noise is a signficiant source of coherent errors in NISQ applications and
these numerical techniques offer effective avenues to mitigate them.
