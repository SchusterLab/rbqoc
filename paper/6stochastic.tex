\section{Robustness to Time-Dependent Parameter Deviations \label{sec:stochastic}}
An additional source of experimental error arises from time-dependent
Hamiltonian parameter deviations. For many flux-biased and inductively-coupled
superconducting circuit elements, magnetic flux noise is a significant
source of coherent errors. Magnetic flux noise
modifies the fluxonium Hamiltonian \eqref{eq:hamiltonian}
by $a(t) \gets a(t) + \delta a(t)$.
The spectral density of $\delta a(t)$ follows a
1/$f$ distribution for a range of devices, consisting primarily of low frequency
noise \cite{bialczak20071f, koch2007model,
  yoshihara2006decoherence, yoshihara2010correlated}.
Analytic methods to combat flux noise
take advantage of the low-frequency characteristic and
treat the noise as quasi-static, performing generalizations of the spin-echo technique
to compensate for erroneous drift. The analytic gate considered here
follows this strategy.

We task the analytic gate and those produced by
the robust control methods of the previous section
to realize an $X/2$ gate subject to 1/$f$ flux noise.
We compute gate errors for each method
by evolving system under the Hamiltonian in \eqref{eq:hamiltonian}
where the optimized flux amplitude is modified $a(t) \gets a(t) + \delta a(t)$.
The flux noise $\delta a(t)$ is generated by
filtering white noise sampled from a standard normal distribution with a finite
impulse response filter \cite{saspweb2011}.
It is then scaled by the 
flux noise amplitude of our device $A_{\Phi} = 5.21 \mu \Phi_{0} \implies
\delta a (t) \sim \sigma_{a} = 2.5 \cdot 10^{-5} \textrm{GHz}$.
The unscented sampling method is modified so that its samples
are subject to 1/$f$ flux noise by carrying the state of a finite impulse response filter
in the augmented state vector \eqref{eq:astatecontrols}.
In principle, we could modify the sampling method
similarly but we choose to sample statically at $\sigma_{a}$ for comparison.
The derivative methods require no algorithmic modification from the static case,
but the TDSE is now differentiated with respect to $a(t)$ instead of $f_{q}$ as
in \eqref{eq:d1dyn}.

We simulate successive applications of the gate constructed by each method
and compute the cumulative gate error
after each application, see Figure \ref{fig:stochastic}. Both the analytic
and numerical methods achieve single gate errors
sufficient for quantum error correction.
Despite converging on qualitatively different solutions, the
numerical methods perform similarly in the concatenated
gate application comparison. They achieve a two
order of magnitude cumulative gate error reduction over the analytic method after $200$
gate applications $\sim 11 \mu\textrm{s}$.
1/$f$ flux noise is a significiant source of coherent errors in NISQ applications and
these numerical techniques offer effective avenues to mitigate them.
