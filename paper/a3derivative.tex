\section{Derivative Method}
\label{appendix:derivative}
We comment on the optimization metrics of the derivative methods and then
outline how to efficiently integrate their dynamics.
The 1\textsuperscript{st}-order derivative method tends
to outperform the 2\textsuperscript{nd}-order
derivative method in the small, static detuning regime, see Figure \ref{fig:static}b.
The norms of the state derivatives for the 1\textsuperscript{st}- and 2\textsuperscript{nd}-order
methods are provided in Table \ref{tab:dnorm} for the solutions at $t_{N} = 60$ns.
The 2\textsuperscript{nd}-order method is able to decrease the 2\textsuperscript{nd}-order
state derivate norm relative to the 1\textsuperscript{st}-order method at the expense of increasing
its 1\textsuperscript{st}-order derivative norm. In the problem
we have studied it is likely the 2\textsuperscript{nd}-order derivative norm has a smaller
contribution to the gate error than the 1\textsuperscript{st}-order derivative norm. A
careful analysis could be completed for future problems to predict the efficacy of the
derivative method at each order.

\begin{table}[h]
  \begin{tabular}{c | c | c}
    Method & ${\lvert \braket{\partial_{f_{q}} \psi_{N} | \partial_{f_{q}} \psi_{N}} \rvert}^{2}$ ($10^{3}$)
    & ${\lvert \braket{\partial^{2}_{f_{q}} \psi_{N} | \partial^{2}_{f_{q}} \psi_{N}} \rvert}^{2}$ ($10^{6}$)\\
    \hline
    D-1 & 0.436 & 57.817\\
    D-2 & 1.702 & 9.030\\
  \end{tabular}
  \caption{Norm of state derivates with respect to the qubit frequency
    for $Z/2$ gates optimized using the derivative methods. The norms are computed
    at the end of the gate's duration $t_{N} = 60$ns and are averaged over each initial state.}
  \label{tab:dnorm}
\end{table}

The dynamics for the derivative methods can be integrated efficiently using exponential integrators
\cite{berland2005solving, einkemmer2017performance}.
General exponential integrators break the dynamics into a linear term and a non-linear term. For example, consider
integrating the dynamics of the first state derivative $\frac{d}{dt} \ket{\partial_{\lambda} \psi} =
H \ket{\partial_{\lambda} \psi} + (\partial_{\lambda} H) \ket{\psi}$ in units of $i\hbar = 1$.
The linear term is $L = H$ and the non-linear term is $N = (\partial_{\lambda} H) \ket{\psi}$.
With zero-order hold on the controls the exact propagation is
\begin{equation}
  \label{eq:dgeneralexp}
  \begin{aligned}
    \ket{\partial_{\lambda} \psi_{k + 1}} &= \exp(\Delta t_{k} L_{k}) \ket{\partial_{\lambda} \psi_{k}}\\
    &+ \int_{0}^{\Delta t_{k}} \exp((\Delta t_{k} - t^{'})L_{k}) N(t_{k} + t^{'}) dt^{'}\\
  \end{aligned}
\end{equation}
General exponential integrators proceed by breaking the integral in \eqref{eq:dgeneralexp}
into a discrete sum, similar to the procedure
for Runge-Kutta schemes. We use a simple approximation known as the Lawson-Euler
method \cite{berland2005solving}
\begin{equation}
  \begin{aligned}
    \ket{\partial_{\lambda} \psi_{k + 1}} &\approx \exp(\Delta t_{k}L_{k}) \ket{\partial_{\lambda} \psi_{k}}\\
    &+ \exp(\Delta t_{k}L_{k}) N_{k}\\
  \end{aligned}
\end{equation}
This method provides a good tradeoff between accuracy and efficiency, requiring one unique matrix
exponential computation per stage. Integration accuracy is not of the utmost importance because the
state derivatives guide the optimization, and do not correspond to experimental parameters
which must be realized with high accuracy.




