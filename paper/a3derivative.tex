\section{Derivative Method}
\label{appendix:derivative}
We comment on the relative performance of the derivative methods and then
outline how to efficiently integrate their dynamics.
We believe the 1\textsuperscript{st}-order derivative method tends
to outperform the 2\textsuperscript{nd}-order
derivative method in the small, static detuning regime, see Figure 2b,
because second-order contributions to the gate error are small.
The norm of the state derivatives for the 1\textsuperscript{st}- and 2\textsuperscript{nd}-order
derivative methods are provided in Table \ref{tab:dnorm} for the solutions of the optimization outlined
in section V at the gate duration $t_{N} = 38$ns.

%%TODO: the values in this table are not up to date until this comment is removed
\begin{table}[h]
  \label{tab:dnorm}
  \begin{tabular}{c | c | c}
    Method & ${\lvert \braket{\partial_{f_{q}} \psi_{N} | \partial_{f_{q}} \psi_{N}} \rvert}^{2}$ ($10^{3}$)
    & ${\lvert \braket{\partial^{2}_{f_{q}} \psi_{N} | \partial^{2}_{f_{q}} \psi_{N}} \rvert}^{2}$ ($10^{7}$)\\
    \hline
    D-1 & 2.051 & 17.935\\
    D-2 & 2.511 & 7.628\\
  \end{tabular}
  \caption{Norm of the first basis state's derivate with respect to the qubit frequency
    for $Z/2$ gates optimized using the derivative methods. The norms are computed
    at the end of the gate's duration $t_{N} = 38$ns.}
\end{table}

The dynamics for the derivative methods can be integrated efficiently using exponential integrators
\cite{berland2005solving, einkemmer2017performance}.
General exponential integrators break the dynamics into a linear term and a non-linear term. For example, consider
integrating the dynamics of the first state derivative $\frac{d}{dt} \ket{\partial_{\lambda} \psi} =
H \ket{\partial_{\lambda} \psi} + (\partial_{\lambda} H) \ket{\psi}$ in units of $i\hbar = 1$.
The linear term is $L = H$ and the non-linear term is $N = (\partial_{\lambda} H) \ket{\psi}$.
With zero-order hold on the controls the exact propagation is
\begin{equation}
  \label{eq:dgeneralexp}
  \begin{aligned}
    \ket{\partial_{\lambda} \psi_{k + 1}} &= \exp(\Delta t_{k} L_{k}) \ket{\partial_{\lambda} \psi_{k}}\\
    &+ \int_{0}^{\Delta t_{k}} \exp((\Delta t_{k} - t^{'})L_{k}) N(t_{k} + t^{'}) dt^{'}\\
  \end{aligned}
\end{equation}
General exponential integrators proceed by breaking the integral in \eqref{eq:dgeneralexp}
into a discrete sum, similar to the procedure
for Runge-Kutta schemes. We use a simple approximation known as the Lawson-Euler
method \cite{berland2005solving}
\begin{equation}
  \begin{aligned}
    \ket{\partial_{\lambda} \psi_{k + 1}} &\approx \exp(\Delta t_{k}L_{k}) \ket{\partial_{\lambda} \psi_{k}}\\
    &+ \exp(\Delta t_{k}L_{k}) N_{k}\\
  \end{aligned}
\end{equation}
This method provides a good tradeoff between accuracy and efficiency, requiring one unique matrix
exponential computation per stage. Integration accuracy is not of the utmost importance because the
state derivatives guide the optimization, and do not correspond to experimental parameters
which must be realized with high accuracy.




