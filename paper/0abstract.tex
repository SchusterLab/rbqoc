%% TITLE
\title{Robust Quantum Optimal Control with Trajectory Optimization}

\author{Thomas Propson}
\email{tcpropson@uchicago.edu}
\affiliation{
  James Franck Institute, University of Chicago, Chicago, Illinois 60637, USA
}
\affiliation{
  Department of Physics, University of Chicago, Chicago, Illinois 60637, USA
}
\author{Brian E. Jackson}
\author{Zachary Manchester}
\affiliation{
  Robotics Institute, Carnegie Mellon University, Pittsburgh, Pennsylvania 15213, USA
}
\author{David I. Schuster}
\affiliation{
  James Franck Institute, University of Chicago, Chicago, Illinois 60637, USA
}
\affiliation{
  Department of Physics, University of Chicago, Chicago, Illinois 60637, USA
}
\affiliation{
  Pritzker School of Molecular Engineering, University of Chicago, Chicago, Illinois 60637, USA
}

\date{\today}

%% ABSTRACT
\begin{abstract}
  The ability to engineer high-fidelity gates on quantum processors in the presence of
  systematic errors remains the primary barrier to achieving quantum advantage.
  Quantum optimal control methods have proven effective in experimentally
  realizing high-fidelity gates, but they require exquisite calibration to be performant.
  We apply robust trajectory optimization techniques to suppress gate errors arising from system
  parameter uncertainty.
  We propose a derivative-based approach that maintains
  computational efficiency by using forward-mode differentiation.
  Additionally, the effect of depolarization on a gate is typically modeled by
  integrating the Lindblad master equation,
  which is computationally expensive.
  We employ a computationally efficient model
  and utilize time-optimal control to achieve high-fidelity gates in the presence of depolarization.
  We apply these techniques to a fluxonium qubit and suppress
  simulated gate errors due to parameter uncertainty below $10^{-7}$ for
  static parameter deviations on the order of $1\%$.
\end{abstract}

\maketitle
