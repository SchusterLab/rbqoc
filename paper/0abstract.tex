%% TITLE
\title{Robust Control of a Fluxonium Qubit}

\author{Thomas Propson}
\email{tcpropson@uchicago.edu}
\affiliation{
  James Franck Institute, University of Chicago, Chicago, Illinois 60637, USA
}
\affiliation{
  Department of Physics, University of Chicago, Chicago, Illinois 60637, USA
}
\author{Brian Jackson}
\author{Zachary Manchester}
\affiliation{
  Robotics Institute, Carnegie Mellon University, Pittsburgh, Pennsylvania 15213, USA
  % Department of Aeronautics and Astronautics Engineering, Stanford University, 496 Lomita Mall, Stanford, CA 94305
}
\author{David I. Schuster}
\affiliation{
  James Franck Institute, University of Chicago, Chicago, Illinois 60637, USA
}
\affiliation{
  Department of Physics, University of Chicago, Chicago, Illinois 60637, USA
}
\affiliation{
  Pritzker School of Molecular Engineering, University of Chicago, Chicago, Illinois 60637, USA
}

\date{\today}

%% ABSTRACT
\begin{abstract}
  %% - 1 + 2 topic importance
  %% - 1 what I have done
  %% - few sentences about primary results
  The ability to engineer high-fidelity gates on quantum processors in the presence of
  systematic errors and decoherence remains the primary challenge requisite to achieving quantum advantage.
  Quantum optimal control techniques have proven effective in experimentally
  realizing high-fidelity gates, but they require exquisite calibration to be performant.
  We apply robust trajectory optimization techniques to suppress gate errors arising from system
  parameter deviations and noise.
  We propose a method that takes advantage of deviant parameter derivative information while maintaining
  computational efficiency by utilizing mixed-mode differentiation.
  Additionally, completely modeling decoherence effects due to longitudinal relaxation requires
  integrating the Lindblad master equation, which is computationally expensive.
  We propose a computationally efficient metric
  and utilize time-optimal control to achieve high fidelity gates in the presence of longitudinal relaxation.
  We demonstrate these techniques numerically on a fluxonium qubit with realistic
  experimental parameters and constraints,
  achieving orders of magnitude gate error reductions from our baseline gate set.
\end{abstract}
