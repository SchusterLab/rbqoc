%% TITLE
\title{Robust Quantum Optimal Control}

\author{Thomas Propson}
\email{tcpropson@uchicago.edu}
\affiliation{
  James Franck Institute, University of Chicago, Chicago, Illinois 60637, USA
}
\affiliation{
  Department of Physics, University of Chicago, Chicago, Illinois 60637, USA
}
\author{Brian Jackson}
\author{Zachary Manchester}
\affiliation{
  Robotics Institute, Carnegie Mellon University, Pittsburgh, Pennsylvania 15213, USA
}
\author{David I. Schuster}
\affiliation{
  James Franck Institute, University of Chicago, Chicago, Illinois 60637, USA
}
\affiliation{
  Department of Physics, University of Chicago, Chicago, Illinois 60637, USA
}
\affiliation{
  Pritzker School of Molecular Engineering, University of Chicago, Chicago, Illinois 60637, USA
}

\date{\today}

%% ABSTRACT
\begin{abstract}
  The ability to engineer high-fidelity gates on quantum processors in the presence of
  systematic errors remains the primary challenge requisite to achieving quantum advantage.
  Quantum optimal control methods have proven effective in experimentally
  realizing high-fidelity gates, but they require exquisite calibration to be performant.
  We apply robust trajectory optimization techniques to suppress gate errors arising from system
  parameter uncertainty.
  We propose a method that takes advantage of uncertain parameter
  derivative information while maintaining
  computational efficiency by transforming high-order differential equations to coupled,
  first-order ODEs.
  Additionally, the effect of depolarization on a gate is most accurately modeled by
  integrating the Lindblad master equation,
  which is computationally expensive.
  We propose a computationally efficient model
  and utilize time-optimal control to achieve high fidelity gates in the presence of depolarization.
  We apply these techniques to a fluxonium qubit with realistic
  parameters and constraints,
  achieving orders of magnitude gate-error reductions from our baseline gate set in simulations.
\end{abstract}

\maketitle
