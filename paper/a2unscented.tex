\section{Unscented Transformation}
\label{appendix:unscented}
In this section, we outline the full unscented sampling procedure.
We consider a state $\psi \in \mathbb{R}^{2n}$ with
a deviant parameter $\lambda \in \mathbb{R}^{d}$ and dynamics
$\psi_{k + 1} = f(\psi_{k}, \lambda_{k})$.
The nominal initial state is given by $\psi_{0}$ with an associated
positive-definite covariance matrix $P_{0} \in \mathbb{R}_{++}^{2n \times 2n}$
which describes the uncertainty in the initial state.
$P_{0}$ is typically non-zero even if the state preparation error is negligible.
The deviant parameter
has zero-mean and its distribution is given by the covariance matrix
$L_{k} \in \mathbb{R}_{++}^{d \times d}$ at knot point $k$. The zero mean assumption
is convenient for deriving the update procedure. A non-zero mean can be encoded
in the dynamics.

The initial $4n + 2d$ sigma points and initial $4n + 2d$ deviant parameters are sampled
from the initial distributions,
\begin{equation}\label{eq:uupdate}
  \begin{bmatrix} \Psi_{0}^{i} \\ \Lambda_{0}^{i} \end{bmatrix} =
  \begin{bmatrix} \bar{\Psi}_{0} \\ 0\end{bmatrix}
    \pm \beta \sqrt{\begin{bmatrix} P_{0} & 0\\ 0 & L_{0}\end{bmatrix}}^{\; i}
\end{equation}
We have written $\bar{\Psi}_{0} = \psi_{0}$. $\beta$
is a hyperparameter which controls the spacing of the covariance contour.
The $(\pm)$ is understood to take $(+)$ for $ i \in \{1, \dots, 2n + d\}$ and $(-)$ for
$i \in \{2n + d + 1, \dots, 4n + 2d\}$. We use the Cholesky factorization
to compute the square root of the
joint covariance matrix, though other methods
such as the principal square root may be employed.
The superscript on the matrix square root indicates the $i$\textsuperscript{th}
column (mod $2n + d$) of the lower triangular Cholesky factor.
Then, the sigma points are normalized,
\begin{equation}\label{eq:unormalize}
  \Psi_{0}^{i} \gets \frac{\Psi_{0}^{i}}{\sqrt{\Psi_{0}^{i^{T}} \Psi_{0}^{i}}}
\end{equation}
The sigma points are propagated to the next knot point,
\begin{equation}\label{eq:upropagate}
  \Psi_{1} = f(\Psi_{0}, \Lambda_{0})
\end{equation}
The mean and covariance of the sigma points are computed,
\begin{align}
  \bar{\Psi}_{1} &= \frac{1}{4n + 2d} \sum_{i = 1}^{4n + 2d} \Psi_{1}^{i}\\
  P_{1} &= \frac{1}{2 \beta^{2}} \sum_{i = 1}^{4n + 2d}
  (\Psi^{i}_{1} - \bar{\Psi}_{1})^{T}(\Psi^{i}_{1} - \bar{\Psi}_{1})
\end{align}
The sigma points are then resampled and propagated to the next knot point using
\eqref{eq:uupdate}, \eqref{eq:unormalize}, and \eqref{eq:upropagate}. Our
choice of sigma points follows the prescription in equation 11 of \cite{julier2004unscented}.
Prescriptions that require fewer sigma points exist \cite{julier2002reduced}.




