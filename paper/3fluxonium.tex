\section{QOC on the Fluxonium \label{sec:fluxonium}}
In the following we study
the quantum optimal control problem on the fluxonium qubit.
The fluxonium qubit is a promising building block for superconducting
circuits, and the accurate
two-level approximation of its Hamiltonian makes
quantum optimal control on a classical computer inexpensive.
In the two-level
approximation the Hamiltonian takes the form:
\begin{align}
  H/h &= f_{q} \frac{\sigma_{z}}{2} + a(t) \frac{\sigma_{x}}{2}
  \label{eq:hamiltonian}
\end{align}
where $f_{q} = 14$MHz is the qubit frequency at the flux frustration point,
$a(t)$ is the flux drive amplitude, $h$ is Planck's constant, and $\sigma_{x}, \sigma_{y}$
are Pauli matrices. We consider the task of constructing $Z/2$, $Y/2$, and $X/2$
gates for the fluxonium qubit subject to experimental constraints, decoherence, and
Hamiltonian parameter deviations. We compare the gates we obtain with numerical
methods to the analytically constructed gates reported in
\cite{zhang2020universal} for the same device.

The optimization problem takes the form:
\begin{mini!}[2]
  {x_{0:N}, u_{0:N\text{-}1}}{\sum_{k=1}^N \norm{x_k\text{-}x_f}_{Q_k}
    + \sum_{k=1}^{N-1} \norm{u_k}_{R_k}}{}{} \label{eq:costfun}
    \addConstraint{x_{k+1}}{= f(x_k, u_k)}  \label{eq:dyn_con}
    \addConstraint{(x_1, x_N \; \textrm{given})}
    \addConstraint{g_k(x_k,u_k)}{\leq 0}
    \addConstraint{h_k(x_k,u_k)}{=0}
\end{mini!}
where the augmented state and controls are given by:
\begin{equation}
  x = \begin{bmatrix} \psi^{(1)} \\ \psi^{(2)} \\ \int a \ dt\\ a \\ \pdv*{a}{t} \end{bmatrix} \quad
  u = \begin{bmatrix} \pdv*[2]{a}{t} \end{bmatrix}
  \label{eq:astatecontrols}
\end{equation}
The initial and final conditions are determined by the
desired gate, which requires two quantum states
to specify $\psi^{(1)}$, $\psi^{(2)}$.
The ALTRO implementation we use does not currently support complex numbers so
we compute in the isomorphism $\mathcal{H}(\mathbb{R}^{2n}) \cong \mathcal{H}(\mathbb{C}^{n})$
given in \cite{leung2017speedup}:
\begin{equation}
  H \psi \cong \begin{bmatrix} H_{\textrm{re}} & -H_{\textrm{im}} \\ H_{\textrm{im}} & H_{\textrm{re}}\end{bmatrix}
  \begin{bmatrix} \psi_{\textrm{re}} \\ \psi_{\textrm{im}}\end{bmatrix}
  \label{eq:isomorphism}
\end{equation}
The discrete dynamics function \eqref{eq:dyn_con} integrates the TDSE dynamics for the quantum states
using \eqref{eq:tdse} and \eqref{eq:hamiltonian}
and integrates the moments of the flux drive amplitude. Exposing lower order moments of the flux
drive amplitude allows us to penalize their norms, smoothing the flux drive amplitude
and mitigating AWG ringing due to high frequency transitions.
The matrices $Q_{k}$ and $R_{k}$ define the penalty metric.
Including objectives at each knot point smoothens the optimization landscape, though
the importance of the final state objective is encoded in the relative
weight $Q_{N} \sim N \cdot Q_{k}$.
We choose $Q$ to be diagonal because it is computationally efficient. This corresponds
to penalizing phase differences between the quantum states, although
other phase-insensitive metrics such as infidelity may be employed.


In addition to imposing optimization constraints that
reflect physical limitations of the apparatus, we impose
constraints that improve the experimental realization of the control pulse.
To ensure gates may be concatenated arbitrarily without
inducing AWG ringing due to high-frequency transitions,
we require $a(t = 0) = a(t = t_{N}) = 0$.
Furthermore, we require $\int_{0}^{t_{N}} a(t) dt = 0$. This
constraint ensures the pulse has zero net flux, mitigating
the hysteresis ubiquitous in flux bias lines,
\cite{battistel2019fast, krantz2019quantum, zhang2020universal}.
We require $-0.5 \textrm{GHz} \le a(t) \le 0.5 \textrm{GHz}$
to ensure the two-level approximation \eqref{eq:hamiltonian}
remains valid. Additionally, we require that each gate achieves
the desired state transition $\psi_{N} = \psi_{f}$.
Both the zero net flux and target quantum state constraint
are then handled by ensuring the target augmented state is
reached $x_{N} = x_{f}$, the constraint function
takes the form
$c_{k} = (x_{k} - x_{f})^{T}(x_{k} - x_{f})$.
The equality and inequality constraints on $a$ are handled
with a bound constraint which takes the form
$c_{k} = (x_{k} - b)^{T}(x_{k} - b)$ if $\lvert x_{k} \rvert \ge b$
and $0$ otherwise.
In addition, the norm of the quantum states
are constrained to $1$ to ensure discretization error is not
exploited $c_{k} = x_{k}^{T}x_{k} - 1$. The selection
of subsets from $x_{k}$ is implied in
the constraint equations.
