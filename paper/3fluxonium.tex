\section{QOC for the Fluxonium \label{sec:fluxonium}}
In the following, we optimize quantum gates
for the superconducting fluxonium qubit--a promising
building block for quantum computers due to its high
coherence times
\cite{earnest2018realization, lin2018demonstration,
  manucharyan2009fluxonium, nguyen2019high,
  zhang2020universal}.
In this section, we use the trajectory optimization
formalism \eqref{eq:gcostfun}-\eqref{eq:ineqcon}
to define the problem \eqref{eq:costfun}-\eqref{eq:ic_con},
which we extend in subsequent sections to account
for experimental error channels.
To high accuracy, we approximate the Hamiltonian near the flux-frustration
point as a two-level system:
\begin{align}
  H/h &= f_{q} \frac{\sigma_{z}}{2} + a(t) \frac{\sigma_{x}}{2},
  \label{eq:hamiltonian}
\end{align}
where $f_{q}$ is the qubit frequency at the flux-frustration point,
$a(t)$ is the flux offset from the flux-frustration point,
$h$ is Planck's constant, and $\sigma_{z}, \sigma_{x}$
are Pauli matrices. We optimize $X/2$,
$Y/2$, and $Z/2$ gates (unitary transformations)
for the fluxonium presented in \cite{zhang2020universal},
and compare them to the analytically constructed gates for
that device.

First, we outline the constraints for the fluxonium gate problem.
All gates presented in this work satisfy these constraints to
a maximum violation of $\sim 10^{-8}$.
Casting this problem as a multi-state transfer, the initial conditions on
the states are $\ket*{\psi^{0}_{1}} = \ket*{0}$, $\ket*{\psi^{1}_{1}} = \ket*{1}$
\eqref{eq:istate_con}
where the superscript is an index $i \in \{0, 1\}$,
and the subscript indicates the knot point $k = 1$.
The states at the final knot point are constrained to be
the target states $\ket*{\psi^{i}_{N}} = \ket*{\psi^{i}_{f}} \equiv U
\ket*{\psi^{i}_{1}} \ \forall \ i$
\eqref{eq:tstate_con} where $U = X/2, Y/2, Z/2$ is the desired gate.
Furthermore, we impose the normalization constraint
${\lvert \braket*{\psi^{i}_{k}}{\psi^{i}_{k}} \rvert}^{2} = 1 \ \forall \ i,k$
\eqref{eq:statenorm_con}
to ensure the solver does not take advantage of discretization errors in numerical integration.
To refer to the discrete moments of the flux, we introduce the notation
$\int^{t_{k}}_{t_{1}} a(t) \ \mathrm{d}t \equiv \int_{t} a_{k}$,
$a(t_{k}) \equiv a_{k}$,
$\pdv*[n]{a(t)}{t} \lvert_{t = t_{k}} \equiv \partial^{n}_{t} a_{k}$.
We impose the zero net flux constraint $\int_{t} a_{N} = 0$
\eqref{eq:znf_con}
which mitigates the inductive drift ubiquitous in flux-bias lines
\cite{battistel2019fast, krantz2019quantum, zhang2020universal}.
The flux is constrained by $\lvert a_{k} \rvert \leq 0.5 \ \textrm{GHz} \ \forall \ k$
\eqref{eq:amp_con}.
Above $0.5$ GHz, we observe population beyond the first two levels, disallowing the
approximation \eqref{eq:hamiltonian}.
We also enforce the boundary condition $a_{1} = a_{N} = 0$ \eqref{eq:bound_con}
so the gates may be concatenated arbitrarily. Additionally,
we have the initial condition $\int_{t} a_{1} = \partial_{t} a_{1} = 0$
\eqref{eq:ic_con}.

The augmented control and augmented state are:
\begin{equation}
  \begingroup
  \renewcommand*{\arraystretch}{1.3}
  u_{k} = \begin{bmatrix} \partial^{2}_{t} a_{k} \end{bmatrix}, \quad
  x_{k} = \begin{bmatrix} \ket{\psi^{0}_{k}} \\ \ket{\psi^{1}_{k}}
    \\ \int_{t} a_{k} \\ a_{k} \\ \partial_{t} a_{k} \end{bmatrix}.
  \endgroup
  \label{eq:astatecontrols}
\end{equation}
The time evolution of these variables are encoded
in coupled, first-order, differential equations
which are integrated in the discrete dynamics function \eqref{eq:dyn_con}.
We integrate the states according to the TDSE \eqref{eq:tdse} and the
fluxonium Hamiltonian \eqref{eq:hamiltonian}.
The ALTRO implementation we use does not currently
support complex numbers, so we repesent the states
in the isomorphism $\mathcal{H}(\mathbb{C}^{n})
\cong \mathcal{H}(\mathbb{R}^{2n})$ given in \cite{leung2017speedup},
\begin{equation}
  H \ket{\psi} \cong \begin{bmatrix} H_{\textrm{re}} & -H_{\textrm{im}}
    \\ H_{\textrm{im}} & H_{\textrm{re}}\end{bmatrix}
  \begin{bmatrix} \ket{\psi}_{\textrm{re}} \\ \ket{\psi}_{\textrm{im}}\end{bmatrix}.
  \label{eq:isomorphism}
\end{equation}

The cost function at each knot point is
$\ell_{k}(x_{k}, u_{k}) = (x_{k} - x_{f})^{T} Q_{k} (x_{k} - x_{f}) + u^{T}_{k} R_{k} u_{k}$
where $Q_{k}$ and $R_{k}$ are positive-definite diagonal matrices we supply.
The $Q_{k}$ term
penalizes deviations from the target augmented state $x_{f}$,
which is given by the constraints we have imposed on
$\ket*{\psi^{i}_{N}}$, $\int_{t} a_{N}$, and $a_{N}$ in addition to
$\partial_{t} a_{f} = 0$.
The $R_{k}$ term penalizes the norm of $\partial^{2}_{t} a_{k}$,
smoothing the flux to mitigate high-frequency AWG transitions.
Stated succinctly, the optimization problem takes the form:
\begin{mini!}[2] 
  {x_{1:N}, u_{1:N\text{-}1}}{\sum_{k=1}^N {(x_k-x_f)}^{T} Q_k (x_k-x_{f})
    + \sum_{k=1}^{N-1} {u_k}^{T} R_k u_{k}}{}{} \label{eq:costfun}
  \addConstraint{x_{k+1}}{= f(x_k, u_k) \ \forall \ k}  \label{eq:dyn_con}
  \addConstraint{\ket{\psi^{0}_{1}} = \ket{0}, \ket{\psi^{1}_{1}} = \ket{1}} \label{eq:istate_con}
  \addConstraint{\ket{\psi^{i}_{N}} = \ket{\psi^{i}_{f}}
    \ \forall \  i} \label{eq:tstate_con}
  \addConstraint{{\lvert \braket{\psi^{i}_{k}}{\psi^{i}_{k}}
      \rvert}^{2} = 1 \ \forall \ i, k} \label{eq:statenorm_con}
  \addConstraint{{\textstyle \int_{t}} a_{N} = 0} \label{eq:znf_con}
  \addConstraint{|a_{k}| \leq 0.5 \ \textrm{GHz} \ \forall \ k} \label{eq:amp_con}
  \addConstraint{a_{1} = a_{N} = 0} \label{eq:bound_con}
  \addConstraint{{\textstyle \int_{t}} a_{1} = \partial_{t} a_{1} = 0}. \label{eq:ic_con}
\end{mini!}

Next, we remark on our problem formulation.
We penalize the squared error of the final state and the target
state \eqref{eq:costfun}, rather than their infidelity,
because the Hessian of the squared error cost function is diagonal--which
makes matrix multiplications fast--and we wish to optimize $Z/2$ gates,
which requires a metric that is sensitive to global phases for the initial
states $\ket{0}$ and $\ket{1}$.
We penalize the squared error at all knot points
because it benefits the iLQR solving stage \cite{Jackson2020altroc},
not to incentivize early achievement
of the desired gate \cite{leung2017speedup}.

In addition to the target state cost function, we also
impose the target state constraint \eqref{eq:tstate_con}, which
says that the final state must match the target state, including its global phase,
up to our chosen maximum constraint violation $\sim 10^{-8}$.
If we did not impose this constraint, the optimizer would be
allowed to sacrifice the closed-system gate error to achieve better
performance on the depolarization or robustness cost functions we will
introduce in the following sections, which is undesirable.
To enforce a constraint in standard QOC frameworks,
the prefactor for the constraint function is manually increased
between separate optimization instances until the constraint is satisfied
\cite{heeres2017implementing, leung2017speedup, reinhold2019controlling},
which becomes infeasible for more than one constraint.
ALM automates these prefactor updates to find
a solution that satisfies all of the constraints.
Hence, ALTRO's ability to handle multiple constraints makes it
an attractive solver for robust QOC problems.

In extraordinarily difficult cases of
QOC \cite{abdelhafez2020universal}, it may be impossible
to obey the physics of the system and achieve the desired gate,
equivalently, the dynamics constraint \eqref{eq:dyn_con}
and the target state constraint \eqref{eq:tstate_con} may be mutually unsatisfiable.
In this case, the prefactors for the constraint function terms
in the ALM objective will tend to infinity--leading to numerical instability--and the
optimization will not converge. To maintain a constrained approach in this situation,
the maximum constraint violation for the target state constraint can be raised
to a level commensurate with the minimum acceptable gate error.

Finally, for the indirect ALM-iLQR solving stage,
the state at each knot point is obtained by integrating
the TDSE, so the dynamics constraint \eqref{eq:dyn_con} is satisfied by construction.
This is not the case for the direct projected Newton stage, where the states
are free parameters that are adjusted to satisfy the TDSE. However,
the solution's deviation from the TDSE is never more than the maximum constraint violation.
When we report gate errors in this text, we explicity integrate the TDSE with the
flux produced by the optimization to ensure accuracy. Exploring the benefit
of direct optimization approaches for QOC is an interesting direction for future work.
