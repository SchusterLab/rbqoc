\section{Depolarization Awareness \label{sec:longitude}}
The strength of depolarization varies
with the control parameters in a range of superconducting circuit platforms,
including the fluxonium. In this section, we outline a method
for tuning the flux amplitude to mitigate depolarization.
Previous work has modeled the gate error due to depolarization
by evolving density matrices under a master
equation \cite{rembold2020introduction, schulteherbruggen2011optimal}
or evolving a large number of states in a quantum trajectory approach
\cite{abdelhafez2019gradient}.
We avoid the increase in computational complexity required for these
techniques by
penalizing the integrated rate (probability) of depolarization in optimization.
Using this probability as a proxy for the gate error incurred
is reasonable because losses due to depolarization increase monotonically
in time. Additionally, for a constant $T_{1}$ time--the $1/e$ depolarization
time for the qubit--a shorter gate duration
would favor a lower depolarization probability. We allow
the optimizer to tune the gate time to minimize the
depolarization probability.

The depolarization probability is given by,
\begin{equation}
  P_{1}(t) = \int_{0}^{t} T_{1}^{-1}(a(t^{\prime})) dt^{\prime}
\end{equation}
This value is appended to the augmented state \eqref{eq:astatecontrols}
and its norm is penalized in the objective \eqref{eq:costfun} by setting
the corresponding element of the final augmented state $x_{f}$ to $0$.
$T_{1}(a_{k})$ is obtained at each knot point by evaluating
a spline fit to experimental data of the form $\{(a, T_{1})\}$,
see Figure \ref{fig:longitudeb}.
It is also possible to fit a spline to theoretically obtained data.
However, $T_{1}$ values are known to fluctuate greatly
with laboratory temperatures \cite{klimov2018fluctuations}.
Interpolating $T_{1}$ from experimental data
increases the fridge truth of the simulation.

We allow the optimizer to tune the gate time by
making the time step between each knot point $\Delta t_{k}$
a decision variable. Promoting $\Delta t_{k}$ to a decision variable, rather
than the number of knot points $N$, preserves the
Markovianity of the trajectory
optimization problem. The square root of the time step $\sqrt{\Delta t_{k}}$
is appended to the augmented control \eqref{eq:astatecontrols}
and the squared root
of the time step $\lvert \Delta t_{k} \rvert$ is used
for numerical integration in the discrete dynamics function \eqref{eq:dyn_con}.
To ensure numerical
integration accuracy is maintained, we constrain
the bounds of the time step at each knot point.

We compare the performance of the $X/2$, $Y/2$, and $Z/2$ gates obtained with
our numerical method to the analytic gates. We
use the Lindblad master equation to simulate $T_{1}$ dissipation for successive
gate applications, and compute the cumulative gate error
after each application, see Appendix \ref{appendix:longitude}.
The flux amplitudes for the numerical gates are similar; the
waveforms are approximately periodic
with amplitudes $\sim 0.2 \textrm{GHz}$, see Figure \ref{fig:longitudea}. \todo{Floquet}.
Their gate times are greater
than their analytic counterparts, but they
reach higher amplitudes and therefore higher $T_{1}$ times.
Although the analytic and numerical gates attain single gate errors sufficient for
quantum error correction ($< 10^{-4}$), which are required for fault-tolerant quantum computing
\cite{aharonov2008fault, fowler2009high, gottesman1997stabilizer},
the single gate errors for the numerical
gates are a factor of $5$ less
than those for the analytic gates, commensurate to the
depolarization probability reductions, see Appendix \ref{appendix:longitude}.
The gate error reported in this text is the infidelity
of the evolved state and the target state averaged over 1000 pseudo-randomly
generated initial states.
The numerical $Z/2$ and $Y/2$ gates produce similar
results in the concatenated gate application comparison,
yielding cumulative gate errors below $8 \cdot 10^{-3}$
over $2000$ gate applications $\sim 40 \mu\textrm{s}$, see Figure \ref{fig:longitudec}.
The cumulative gate error for the numerical $X/2$ gate is
$1.7 \cdot 10^{-2}$ over $2000$ gate applications $\sim 124 \mu\textrm{s}$.
The low cumulative gate errors for high gate counts produced
by the numerical gates are critical for
noisy, intermediate-scale quantum (NISQ) applications.
These improvements are significant for the realistic constraints we have imposed
on the gates, and do not represent a fundamental limit to the optimization methods we have
employed.
