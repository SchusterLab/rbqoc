\section{Depolarization Mitigation\label{sec:longitude}}
In this section, we outline a method
for optimizing the flux to mitigate depolarization.
For many superconducting circuits, the 1/$e$ depolarization time ($T_{1}$) is independent of the control parameters,
so the fastest possible gate incurs the least depolarization error
\cite{schulteherbruggen2011optimal}.
For the fluxonium, however, $T_{1}$ is strongly dependent on the flux.
We enable the optimizer to trade longer gate times
for lower depolarizaiton times or vice-versa by making the gate time a decision variable.
Additionally, previous work has modeled the gate error due to depolarization
by evolving density matrices under a master
equation \cite{rembold2020introduction, schulteherbruggen2011optimal}
or evolving a large number of states in a quantum trajectory approach
\cite{abdelhafez2019gradient}.
We avoid the increase in computational complexity required for these
techniques by penalizing the integrated rate (probability) of depolarization in optimization.
Using this probability as a proxy for the gate error incurred
is reasonable because depolarization errors are incoherent and therefore
increase monotonically in time without interference.

The depolarization probability is given by,
\begin{equation}
  P_{1}(t) = \int_{0}^{t} T_{1}^{-1}(a(t^{\prime})) dt^{\prime}.
\end{equation}
This value is appended to the augmented state \eqref{eq:astatecontrols}
and its norm is penalized in the objective \eqref{eq:costfun} by setting
the corresponding element of the final augmented state to zero.
$T_{1}(a_{k})$ is obtained at each knot point by evaluating
a spline fit to experimental data of the form $\{(a, T_{1})\}$,
see Figure \ref{fig:longitudeb}.

It is also possible to fit a spline to theoretically obtained data.
However, experimental data often deviates from theoretical predictions
because it is difficult to account for all loss channels \cite{zhang2020universal},
and $T_{1}$ values often fluctuate on hour
time scales \cite{klimov2018fluctuations}. (You could cut this - no need to justify using experimental data if you've got it! Maybe just cite where the data is coming from, if published.)

We allow the optimizer to tune the gate time by
making the time step between each knot point $\Delta t_{k}$
a decision variable. Promoting $\Delta t_{k}$ to a decision variable, rather
than the number of knot points $N$, preserves the
Markovianity of the trajectory
optimization problem. The square root of the time step $\sqrt{\Delta t_{k}}$
is appended to the augmented control \eqref{eq:astatecontrols}
and the squared root (this looks like a typo)
of the time step $\lvert \Delta t_{k} \rvert$ is used
for numerical integration in the discrete dynamics function \eqref{eq:dyn_con}.
To ensure numerical
integration accuracy is maintained, we constrain
the bounds of the time step at each knot point.

We analyze the effect of depolarization on
the $X/2$, $Y/2$, and $Z/2$ gates obtained with
our numerical method and the analytic gates. We
use the Lindblad master equation to simulate $T_{1}$ dissipation for successive
gate applications, and compute the cumulative gate error
after each application, see Appendix \ref{appendix:longitude}.
\todo{Jens, Danny, Ziwen, any idea what the correct way to simulate depolarization is
  when $T_{1}$ is time dependent? Jens is going to have the best opinion on this - personally I think what you've done seems correct}.
The gate error reported in this text is the infidelity
of the evolved state and the target state averaged over 1000 pseudo-randomly (what is this a cryptography paper?! "pseudo" lol)
generated initial states.
The flux pulses for the numerical gates
are approximately periodic
with amplitudes $\sim 0.2 \textrm{GHz}$, see Figure \ref{fig:longitudea}.
They are reminiscent of the analytically determined Floquet operations
for a fluxonium described in \cite{huang2020engineering}
and realized in \cite{mundada2020floquet}.
The numerical gate times are greater
than the analytic gate times, but the
numerical flux pulses
spend more time at higher amplitudes, achieving higher $T_{1}$ times.
The single gate errors for both the analytic and numerical gates are
less than $10^{-4}$, which makes them sufficient for quantum error correction--a
prerequisite for fault-tolerant quantum
computing \cite{aharonov2008fault, knill2005quantum, gottesman1997stabilizer}.
However, the numerical gates achieve single gate errors
$\sim 5$ times less than those for the analytic gates,
which tracks closely with their relative improvement
on the depolarization probability metric, see Appendix \ref{appendix:longitude}.
This single gate error advantage corresponds to a significant reduction in error correction resources (this seems like a strong statement - citation?).
Furthermore, for successive gate applications, the gate error due to depolarization is approximately linear
in the gate count, which we expect for $t \ll T_{1}$, see Figure \ref{fig:longitudec}.
The gate error reduction for large gate counts is important for noisy, intermediate-scale quantum (NISQ)
applications. These improvements are significant for the constraints we have imposed on the gates,
and do not represent a fundamental limit to the optimization methods we have employed.

