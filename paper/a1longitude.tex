\section{Depolarization \label{appendix:longitude}}
We comment on the depolarization metrics and then give
our procedure for integrating the Lindblad master equation.
The integrated depolarization rate and the gate error due to
depolarization metrics \textcolor{blue}{[The error is due to depolarization METRICS?]} are compared in Table \ref{tab:longitude} for the
numerical experiment described in Section \ref{sec:longitude}.
The relative performance \textcolor{blue}{[what is relative performance?]} of the analytic and numerical techniques
is similar across the two metrics.

\begin{table}[ht]
  \begin{tabular}{c | c | c | c | c | c | c}
         & $D_{1\textrm{A}}$ & $D_{1\textrm{N}}$ & & GE$_{\textrm{A}}$ & GE$_{\textrm{N}}$ &\\
    Gate & $(10^{-5})$ & $(10^{-5})$ & $D_{1\textrm{A}}/ D_{1\textrm{N}}$ & $(10^{-5})$ &
    $(10^{-5})$ & GE$_{\textrm{A}} / \textrm{GE}_{\textrm{N}}$\\
    \hline
    Z/2 & 5.745 & 1.149 & 5.000 & 0.888   & 0.185  & 4.791 \\
    Y/2 & 5.253 & 1.157 & 4.540 & 0.770 & 0.186   & 4.132\\
    X/2 & 16.251 & 2.660 & 6.109 & 2.674 & 0.432  & 6.200\\
  \end{tabular}
  \caption{
    Single-gate integrated depolarization rate ($D_{1}$)
    and single-gate error due to depolarization (GE).
    Values are reported for the analytic (A) and numerical (N) gates.
  }
  \label{tab:longitude}
\end{table}

We employ the Lindblad master equation
to compute the gate error due to depolarization.
This equation takes the form:
\begin{equation}
  \frac{d}{dt} \rho = -\frac{i}{\hbar} [H, \rho]
  + \sum_{i} \gamma_{i} (L_{i} \rho L_{i}^{\dagger}
  - \frac{1}{2} \{L_{i}^{\dagger} L_{i}, \rho\}),
\end{equation}\textcolor{blue}{[I suggest to remove the rest of the sentence completely; should not state any of this for physics journal]}
where $\rho = \ket{\psi}\bra{\psi}$ \textcolor{blue}{[This is not correct (unless your state were to stay pure for all times)]} is the density matrix,
$[\cdot, \cdot]$ is the algebraic commutator, and $\{\cdot, \cdot \}$ is the algebraic
anti-commutator.
For depolarization, $\gamma_{\pm} = T_{\pm}^{-1}$,
$L_{\pm} = \sigma^{\pm} \equiv (\sigma_{x} \pm i \sigma_{y})/2$.
The depolarization times $T_{+} = T_{-} = 2 T_{1}$ are obtained at each time step
from the spline shown in Figure \ref{fig:longitudeb}.
We obtain the $T_{1}$ values in this spline
by driving the qubit at the desired flux bias
and monitoring the resultant decay. For more details
on these measurements, consult \cite{zhang2020universal}.
Because $T_{1}$ depends on the flux $a(t)$, so do
the decay rates $\gamma_{\pm}$.
Integrating the master equation with time-dependent decay rates
provides a heuristic for how gates might  perform in
the experiment. 
This procedure may not be strictly correct
when decay rates change significantly on the time scale of the relaxation time $\Delta t_{k} \ll T_{1}$,
which is the regime we are operating in. Standard
derivations of the Lindblad master equation do not account for
time-dependent decay rates \cite{manzano2020a}. A more thorough
treatment of this regime in future work would unlock new insights for
quantum computing platforms where decoherence is strongly
dependent on the control parameters.

In order to use exponential integrators, we employ
the vectorization/Choi-Jamiolkowski isomorphism \cite{Landi2018},
\begin{equation}
  \frac{d}{dt} \textrm{vec}({\rho}) = \hat{\mathcal{L}}\, \textrm{vec}({\rho}),
\end{equation}
\begin{equation}
  \begin{aligned}
    \hat{\mathcal{L}} &= -i(\openone \otimes H - H^{T} \otimes \openone)\\
    &+ \sum_{i} \gamma_{i}
    (L_{i}^{*} \otimes L_{i} - \frac{1}{2} (\openone \otimes L_{i}^{\dagger}L_{i}
    - L_{i}^{T}L_{i}^{*} \otimes \openone)),
  \end{aligned}
\end{equation}
where $\rho = \sum_{i, j} \alpha_{ij} \ket{i}\bra{j}$
and $\textrm{vec}(\rho) = \sum_{i, j} \alpha_{ij} \ket{i} \otimes \ket{j}$.
Because $a(t)$ is constant between time steps $[t_{k}, t_{k + 1})$ \textcolor{blue}{[for real, because of the AWG? even that would get filtered, right? or as an approximation because of discretization?]},
so are $H(a(t))$ and $\gamma_{\pm}(a(t))$.
Therefore, the exact solution is
$\textrm{vec}(\rho_{k + 1}) = {\exp}{\textstyle(}\Delta t_{k}
\hat{\mathcal{L}}_{k}{\textstyle)} \textrm{vec}(\rho_{k})$.
The vector isomorphism transforms $(n \times n) \times (n \times n)$
matrix-matrix multiplications to $(n^{2} \times n^{2}) \times n^{2}$ matrix-vector
multiplications. For small $n$ and zero-order hold \textcolor{blue}{[what is zero-order hold?]} on the controls, we find that it is
faster to use an exponential integrator on the vectorized equation than to perform
Runge-Kutta on the unvectorized equation. The latter requires decreasing the interval $\Delta t_{k}$
to maintain accuracy, resulting in more time steps.
