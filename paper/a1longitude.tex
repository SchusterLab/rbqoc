\section{Depolarization \label{appendix:longitude}}
We comment on the depolarization metrics and then give
our procedure for integrating the Lindblad master equation.
The integrated depolarization rate and the gate error due to
depolarization metrics are compared in Table \ref{tab:longitude} for the
numerical experiment described in Section \ref{sec:longitude}.
The relative performance of the analytic and numerical techniques
is similar across the two metrics.

\begin{table}[ht]
  \begin{tabular}{c | c | c | c | c | c | c}
         & $D_{1\textrm{A}}$ & $D_{1\textrm{N}}$ & & GE$_{\textrm{A}}$ & GE$_{\textrm{N}}$ &\\
    Gate & $(10^{-5})$ & $(10^{-5})$ & $D_{1\textrm{A}}/ D_{1\textrm{N}}$ & $(10^{-5})$ &
    $(10^{-5})$ & GE$_{\textrm{A}} / \textrm{GE}_{\textrm{N}}$\\
    \hline
    Z/2 & 5.745 & 1.149 & 5.000 & 1.776 & 0.371 & 4.787\\
    Y/2 & 5.253 & 1.157 & 4.540 & 1.539 & 0.370 & 4.159\\
    X/2 & 16.251 & 2.660 & 6.109 & 5.347 & 0.863 & 6.196\\
  \end{tabular}
  \caption{
    Single gate integrated depolarization rate ($D_{1}$)
    and single gate error due to depolarization (GE).
    Values are reported for the analytic (A) and numerical (N) gates.
  }
  \label{tab:longitude}
\end{table}

We employ the Lindblad master equation
to compute the gate error due to depolarization.
This equation takes the form:
\begin{equation}
  \frac{d}{dt} \rho = \frac{-i}{\hbar} [H, \rho]
  + \sum_{i} \gamma_{i} (L_{i} \rho L_{i}^{\dagger}
  - \frac{1}{2} \{L_{i}^{\dagger} L_{i}, \rho\}),
\end{equation}
where $\rho = \ket{\psi}\bra{\psi}$ is the density matrix,
$[\cdot, \cdot]$ is the algebraic commutator, and $\{\cdot, \cdot \}$ is the algebraic
anti-commutator.
For depolarization $\gamma_{\uparrow} = T_{1, \uparrow}^{-1}$,
$\gamma_{\downarrow} = T_{1, \downarrow}^{-1}$,
$L_{\uparrow} = \sigma^{+}/2$, and
$L_{\downarrow} = \sigma^{-}/2$ where $\sigma^{\pm} = \sigma_{x} \pm i \sigma_{y}$.
Both $T_{1, \uparrow}$ and $T_{1, \downarrow}$ are obtained at each knot point
from the spline shown in Figure \ref{fig:longitudeb}.
We obtain the $T_{1}$ values in this spline
by driving the qubit at the desired flux amplitude
and monitoring the resultant decay. For more details
on these measurements, consult \cite{zhang2020universal}.

So that we may use exponential integrators, we employ
the Vectorization/Choi-Jamiolkowski isomorphism \cite{Landi2018},
\begin{equation}
  \frac{d}{dt} \textrm{vec}({\rho}) = \hat{\mathcal{L}} \textrm{vec}({\rho}),
\end{equation}
\begin{equation}
  \begin{aligned}
    \hat{\mathcal{L}} &= -i(I \otimes H - H^{T} \otimes I)\\
    &+ \sum_{i} \gamma_{i}
    (L_{i}^{*} \otimes L_{i} - \frac{1}{2} (I \otimes L_{i}^{\dagger}L_{i}
    - L_{i}^{T}L_{i}^{*} \otimes I)),
  \end{aligned}
\end{equation}
where $\rho = \sum_{i, j} \alpha_{i, j} \ket{i}\bra{j}$
and $\textrm{vec}(\rho) = \sum_{i, j} \alpha_{i, j} \ket{i} \otimes \ket{j}$.
We use zero-order hold on the controls--equivalently, $a(t)$ is piecewise-constant
on the intervals $[t_{k}, t_{k + 1}]$. Therefore,
$H(a(t))$, $L(a(t))$, and $\gamma_{i}(a(t))$ are also piecewise-constant.
So, the exact solution is
$\textrm{vec}(\rho_{k + 1}) = {\exp}{\textstyle(}\Delta t_{k} \hat{\mathcal{L}}_{k}{\textstyle)} \textrm{vec}(\rho_{k})$.
This isomorphism transforms $(n \times n) \times (n \times n)$
matrix-matrix multiplications to $(n^{2} \times n^{2}) \times n^{2}$ matrix-vector
multiplications. For small $n$ and zero-order hold on the controls, we find that it is
faster to use an exponential integrator on the vectorized equation than to perform
Runge-Kutta on the unvectorized equation. The latter requires decreasing the integration
time step to maintain accuracy, resulting in more knot points.
