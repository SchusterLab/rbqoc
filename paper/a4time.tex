\section{Computational Performance \label{appendix:time}}
We provide runtimes for our optimizations and comment on the scaling of the
robustness methods. The runtimes for
the base optimization in Section \ref{sec:fluxonium},
the depolarization optimization in Section \ref{sec:longitude},
and the robust optimizations in Section \ref{sec:static}
are presented in Table \ref{tab:time}
for a $Z/2$ gate at gate times which are multiples of $1/4f_{q} \sim 18$ns.
We performed optimizations on a single core of
an AMD Ryzen Threadripper 3970X 32-Core Processor.
Future work will parallelize the robustness methods using GPUs
\cite{leung2017speedup},
which will enable fast optimizations on high-dimensional Hilbert spaces.

\begin{table}[H]
  \centering
  \begin{tabular} {c | c | c | c }
    & \multicolumn{3}{c}{Average Runtime (s)}\\
    \hline
    $t_{N}$ (ns) & $18$ & $36$ & $72$\\
    \hline
    Base & $0.155 \pm 0.008$ & $7.0 \pm 0.4$ & $15.9 \pm 0.8$\\
    Depol. & $1.69 \pm 0.08$ & - & -\\
    S & $1.77 \pm 0.09$ & $48 \pm 2$ & $280 \pm 10$\\
    U & $75 \pm 4$ & $340 \pm 20$ & $400 \pm 20$\\
    D1 & $6.1 \pm 0.3$ & $27 \pm 1$ & $65 \pm 3$\\
    D2 & $15.7 \pm 0.8$ & $17.3 \pm 0.9$ & $54 \pm 3$\\
  \end{tabular}
  \caption{
    Average runtimes for $Z/2$ optimizations
    using the base, depolarization, sampling (S),
    unscented Sampling (U), and the 1\textsuperscript{st}-
    and 2\textsuperscript{nd}-order derivative methods (D1, D2).
  }
  \label{tab:time}
\end{table}

Now we present the problem size complexities for the robustness methods.
For the sampling method, the size of the augmented state vector
is $O(dn^{3})$, where $d$ is the number of uncertain parameters and
$n$ is the dimension of the Hilbert space. There are $n^{2}$ initial states
in the operator basis, $2d$ sample states per initial state,
and each state has $2n$ real numbers.
For the unscented sampling method, the size of the augmented state vector
is $O(dn^{3} + n^{4})$.
There are $n^{2}$ initial states in the operator basis,
$2(2n + d)$ sample states per initial state,
and each state has $2n$ real numbers.
For the $m$\textsuperscript{th}-order derivative method, the size of the augmented state vector
is $O(dmn^{3})$. There are $n^{2}$
initial states in the operator basis, $dm$ state derivatives per initial state,
and each state has $2n$ real numbers.
