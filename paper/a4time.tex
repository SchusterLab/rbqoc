\section{Computational Performance and Scaling \label{appendix:time}}
We provide runtimes for our optimizations and comment on the scaling of the
robustness methods. The runtimes for the base optimization in Section \ref{sec:fluxonium},
longitudinally aware optimization in Section \ref{sec:longitude},
and robust optimizations in Section \ref{sec:static} are presented
in Table \ref{tab:time} for a $Z/2$ gate at multiples
of the analytic gate time.
\todo{timing table}

\begin{table}[ht]
  \begin{tabular} {c | c | c | c | c | c | c}
    & \multicolumn{6}{c}{Wall Time ($\pm 0.001$ s)}\\
    \hline
    Gate Time (ns) & Base & Long. & S & SU & D1 & D2\\
    \hline
    $18$ & 0.155 & 1.688 & 1.773 & & 16.713 & 48.398\\ % 0.155049
    $36$ & 7.014 & - & 48.213 & & 67.838 & 81.030\\
    $72$ & 15.906 & - & & & &\\
  \end{tabular}
  \caption{Programs were executed on a single thread of an AMD Ryzen Threadripper 3970X 32-Core Processor.}
  \label{tab:time}
\end{table}

\todo{Ask Dave if we should keep complexity analysis in main text.}
Now we present the problem size complexities for the robustness methods.
For the sampling method, the size of the augmented state vector
is $O(dn^{3})$. $d$ is the number of deviant parameters.
$n$ is the dimension of the Hilbert space. There are $n^{2}$ initial states
in the operator basis, $2d$ sample states per initial state,
and each state has $2n$ elements.
For the unscented sampling method, the size of the augmented state vector
is $O(dn^{3} + n^{4})$.
There are $n^{2}$ initial states in the operator basis,
$2(2n + d)$ sample states per initial state,
and each state has $2n$ elements.
For the $m$\textsuperscript{th}-order derivative method, the size of the augmented state vector
is $O(dmn^{3})$. There are $n^{2}$
initial states in the operator basis, $dm$ state derivatives per initial state,
and each state has $2n$ elements.
