\section{Computational Performance \label{appendix:time}}
We provide runtimes for our optimizations and comment on the scaling of the
robustness methods. The runtimes for
the base optimization in Section \ref{sec:fluxonium},
the longitudinally aware optimization in Section \ref{sec:longitude},
and the robust optimizations in Section \ref{sec:static}
are presented in Table \ref{tab:time}
for a $Z/2$ gate at multiples of the analytic $Z/2$ gate time.
Compiler optimizations for statically-sized arrays were utilized
for all methods except for the unscented sampling method.
The unscented sampling method's augmented state vector size
was too large to take advantage of these optimizations,
adversely affecting its run time.
For the robust methods, the runtime scales super-linearly with the
gate time. We expect the runtime to scale linearly with the gate time
for large gate times because
the number of knot points scales linearly with the gate time.
We observe this behavior for the base method, which has a smaller
augmented state vector size than the robust methods.
We performed optimizations on a single CPU thread in this work as a proof
of concept. Future work will parallelize the robustness
methods across the initial states in the operator basis and utilize GPUs
\cite{leung2017speedup},
allowing for fast optimization on large Hilbert spaces.

\begin{table}[ht!]
  \begin{tabular} {c | c | c | c | c | c | c}
    & \multicolumn{6}{c}{Wall Time ($\pm 0.001$ s)}\\
    \hline
    $t_{N}$ (ns) & Base & Long. & S & SU & D1 & D2\\
    \hline
    $18$ & 0.155 & 1.688 & 1.773 & 210.573 & 16.713 & 48.398\\ % 0.155049
    $36$ & 7.014 & - & 48.213 & 4566.236 & 67.838 & 81.030\\
    $72$ & 15.906 & - & 281.372 & 16575.308 & 266.997 & 332.182\\
  \end{tabular}
  \caption{
    Runtimes for $Z/2$ optimizations.
    Programs were executed on a single thread of an
    AMD Ryzen Threadripper 3970X 32-Core Processor.
  }
  \label{tab:time}
\end{table}

\todo{Ask Dave if we should keep complexity analysis in main text.}
Now we present the problem size complexities for the robustness methods.
For the sampling method, the size of the augmented state vector
is $O(dn^{3})$. $d$ is the number of deviant parameters.
$n$ is the dimension of the Hilbert space. There are $n^{2}$ initial states
in the operator basis, $2d$ sample states per initial state,
and each state has $2n$ real numbers.
For the unscented sampling method, the size of the augmented state vector
is $O(dn^{3} + n^{4})$.
There are $n^{2}$ initial states in the operator basis,
$2(2n + d)$ sample states per initial state,
and each state has $2n$ real numbers.
For the $m$\textsuperscript{th}-order derivative method, the size of the augmented state vector
is $O(dmn^{3})$. There are $n^{2}$
initial states in the operator basis, $dm$ state derivatives per initial state,
and each state has $2n$ real numbers.
