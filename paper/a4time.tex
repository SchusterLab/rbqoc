\section{Computational Performance and Scaling}
\todo{timing table}

\todo{make complexity analysis a coherent paragraph}
(sampling) For the sampling method the number of sample states in the augmented state vector
scales as $O(dn^{2})$
because each of $n^{2}$ initial
states is represented by $2d$ samples. $n$ is both the dimension of the Hilbert space
and the size of a state vector. $d$ is the number of deviant parameters.

(unscented) Each initial state from the operator basis
is represented with a distribution of $4n + 2d$ samples. Hence,
the number of states in the augmented state vector scales as $O(n^{3} + dn^{2})$.
The gate error of each sample state is penalized as in the sampling method.


(derivative) For this method $m$ state derivatives are associated with each
initial state from the operator basis.
So, the number of states in the augmented state vector scales as
$O(dmn^{2})$. Additionally, we note that the number of initial
states and samples we have employed for each of the three
methods follows standard perscriptions. However, knowledge
of the problem can typically be used to reduced the number
of required samples.
